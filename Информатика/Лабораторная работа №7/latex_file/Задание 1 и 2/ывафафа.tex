\documentclass[a4paper, 12pt]{article}
\usepackage[utf8]{inputenc}
\usepackage[russian]{babel}

\begin{document}
Сегодня я побывал на виртуальной выставке Эдварда Мунка в Третьяковской гелерее, Лаврушинском переулке. На ней представлено шестьдесят четыре живописные работы, тридцать семь графических, оригинальные фотографии и архивные материалы из его библиотеки. Это лишь маленький кусочек из творческой жизни Эдварда Мунка. Каждая из его работ несет в себе определенный смысл, поэтому, глядя на них, ты начинаешь сопереживать писателю. Начну свой рассказ с того, что мне понравилось на выставке больше всего.

Первой картиной, о которой мне бы захотелось подельться, был автопортрет Эдварда Мунка "В аду". На заднем плане мы можем увидеть страшную черную тень, которая маячит возле писателя. Сама картина выполнена в серо-красных тонах. Мне кажется, что она олицетворяет душевное состояние писателя, окутанное тревогой, на тот момент его жизни. 

Далее я познамился с художественным произведением "Автопортрет с бутылкой вина". На фотографии Мунк изображен очень уставшим. Его руки чуть сжаты и расположены на коленях, как будто он терпеливо ждет, когда ему подадут обед. Писатель не стал проривовывать детали на своем полотне. Вероятно, таким образом он хотел акцентировать внимание зрителей именно на проблеме картины, на том, каким чувства она в себе несет.

Еще одним произведением, которое на меня произвело впечатление, была "Мадонна". В этой картинке писатель воплощает свои собственные переживания и чувства. Глаза девушки полуоткрыты, вероятно, они обозначают усталость. Волосы полураспущены. На этом полотне, как и на многих других, Мунк показывает отчаяние и трагедию человеческой жизни.

Также мне запомнилась картина "Танец Жизни'', написанная Эдвардом Мунком в 1900 году.
 На ней мы можем увидеть танцующих пар мужчин и женщин и двух грусных одиноких девушек. Вероятнее всего они олицетворяют различные периоды жизни. Слева находится молодая девушка, посередине – чуть постарше, а справа – еще старше. Может быть писатель таким образом хотел показать, что нужно пользоваться моментом своей молодости, чтобы затем об этом не сожалеть.

Последним автопортретом Мунка является "Автопортрет между часами и кроватью". На нем мы может видеть уже пожилого писателя. На картине мы можем заметить часы, на которых нет стрелок. Думаю, что автор этим полотнном хотел сказать, что искусство вечно, а человек, руками которого работы писаны, - нет. 

Теперь перейду к пейзажам.
Сейчас мне хочется рассказать о картине Эдварда Мунка "Звездная ночь". Для меня она чем-то похожа на одноименную картину Ван Гога. Скорее всего писатель тоже знал его творчество. Сама картина выполнена в синих цветах, которые придают ей живости. На заднем плане мы можем видеть загадочное звездное небо. От самой картины веет спокойствием. Глядя на нее, чувствуешь духовное умиротворение.

Отдельно хотелось бы выделить картину "Солнце". Симметрично структурированные разноцветные теплых оттенков лучи, исходящие из яркого золотистого центра, занимают все пространство картины. Наш мир зависит от солнца, и поэтому существуют мгновения, когда на серые унылые скалы спускается свет, и на какое-то время меняется цвет жизни.

Рассматривая картины писателя, я получил очень много положительных эмоций. Для меня Мунк – очень сильный человек. Несмотря на свою тяжелую жизнь, он не отчаивался, а продолжал жить дальше. У него есть даже комичная картина на эту тему, которая овеществляет сейчас мною сказанные слова, правда на виртуальной выставке я ее не видел. На том полотне находится Мунк, у которого на столе стоят пустые бутылки алкоголя. Его глаза на картине полны безнадежности и отчаяния.

Кроме картин на выставке есть еще очень много интересного, посвященного писателю. Например, вы можете ознакомиться с содержанием книг, которые находились в библиотеке у Эдварда Мунка. Это поможет вам более подробно понять чувства писателя и то, что он хотел донести до нас в своих произведениях. Кроме того, на выставке присутствуют фотографии писателя. По ним можно определить, чем занимался художник в то время, как он выглядел, с кем был знаком и не только. Помимо всего этого на выставке есть несколько миниатюр, которые писатель нарисовал в годы своей творческой жизни. Все это, несомненно, является настоящим культурным наследием.

Теперь мне бы хотелось рассказать о самом зале, где были расположены произведения Эдварда Муна. На самом деле очень хорошо, что нет почти ничего, отвлекающего от просмотра его работ. Это помогает сосредоточиться на творчестве писателя, и, следовательно, проникнуться всей атмосферой данного места. Зал очень просторный, что дает возможность посмотреть на картины как вблизи, так и издалека. Также, благодаря этому, твой взгляд находится только на одной картине, следовательно, тебя ничего не должно отвлекать. Отдельно хотелось бы выделить освещение. Оно здесь на высочайшем уровне. Там, где нужно, оно присуствует, в других местах его нет. Таким образом создается ощущение загадочности данного места.

Сама по себе Третьяковская галерея очень большая, поэтому, если вам захочется посетить что-нибудь еще, вы сможете сходить в другой зал и посмотреть экспонаты там. В галерее присутствуют произведения искусства от второй половины 19 века почти до наших дней.

Наконец подытожу все, что написано выше. Мне очень понравился данный зал в Третьяковской галерее. Все сделано так, чтобы твое внимание находилось именно на произведениях искусства, а не на окружении. Сама выставка довольно информативная. На ней присутствуют много важных экспонатов со всего периода творческой активности писателя. Кроме картин на ней есть также книги из его библиотеки, фотографии, а также миниатюры.

Из минусов мне не понравилось то, что нельзя узнать названия всех картин, находящихся в данном зале, поэтому приходилось пользоваться интернетом, чтобы их находить, а в брошюре, которую можно было скачать перед посещением музея, хоть и есть несколько названий и картин, но их очень мало. По сравнению с плюсами это, безусловно, не сравнится, хотя все же является небольшим минусом. Может быть я захочу заказать репродукцию какой-либо из картин, расположенных на этой выставке.

Как итог я бы поставил выставке Эдварда Мунка 4.5 балла за полученное удовольствие от времяпрепровождения и качественную проработку зала под экспонаты.	
\end{document}