\subsection*{Определение выпуклого и невыпуклого множества.}

\noindent \textasteriskcentered~Множество является \textit{выпуклым}, если $x, y \in M$, $\lambda \in [0, 1] \Rightarrow \lambda x + (1 - \lambda) y \in M$. В противном 
случае, множество называется \textit{невыпуклым}. Любое подпространство обязательно является выпуклым, однако обратное неверно.

\begin{tikzpicture}
    \draw (0, 0) .. controls (0, 1) and (1, 1) .. (1, 0.5)
                 .. controls (1, 0.25) and (2, 0.25) .. (2, 0.5)
                 .. controls (2, 1) and (3, 1) .. (3, 0.5)
                 .. controls (3, -1) and (0, -1) .. (0, 0);
    \draw (1.5, 0) node{невыпуклое};

    \draw (5, 0) .. controls (5, 1) and (8, 1) .. (8, 0)
                 .. controls (8, -1) and (5, -1) .. (5, 0);

    \draw (6.5, 0) node{выпуклое};


    \draw (12, 0) node{$M$} circle (1);
    \fill (14, 0) node[anchor=north]{$x$} circle[radius = 0.05];
    \draw (14, 0) -- (12, 0);
    \fill (13, 0) node[anchor=south west]{$\hat{x}$} circle[radius = 0.05];
\end{tikzpicture}


\subsection*{Теорема о наилучшем приближении в Н для случая выпуклого, замкнутого множества.\footnote{По идее данная теорема должна была быть в пункте 15.}}

\begin{theorem*}
Пусть $H$ - гильбертовое пространство, $M$ - замкнутое выпуклое в $H$. Тогда $\forall x \in H$ в $M$ $\exists !$(единственный) элемент $\hat{x} : 
\rho(x, M) = \norm{x - \hat{x}}$.
\end{theorem*}

\begin{proof}
\smallskip
\par\noindent \textbullet~Обозначим $d = \rho(x, M) = \inf_{y \in M} \norm{x - y}$. По определению точной нижней грани $\forall n \in \mathbb{N} \; \exists y_n \in M : 
d \le \norm{x - y_n} < d + \dfrac{1}{n}$.

\smallskip 
\noindent \textbullet~Рассмотрим точки $y_n, y_m, x - y_n, x - y_m$ и применим к ним равенство параллелограмма $2 \norm{x - y_n}^2 + 2 \norm{x - y_m}^2 = \norm{2x - y_n 
-y_m}^2 + \norm{y_n - y_m}^2$ - сумма квадратов диагоналей равна сумме квадратов его сторон.

\smallskip 
\noindent \textbullet~Из полученого выше $\norm{2x - y_n - y_m}^2 + \norm{y_n - y_m}^2 < 2(d + \dfrac{1}{n})^2 + 2(d + \dfrac{1}{m})^2$.

\smallskip
\noindent \textbullet~$\norm{2x - y_n - y_m}^2 = 4 \norm{x - \dfrac{y_n + y_m}{2}}^2$. $M$ - выпуклое, $y_n, y_m \in M \Rightarrow \dfrac{y_n + y_m}{2}$ - по выпуклости 
в $M$ будет лежать и их середина, то есть $\norm{x - \dfrac{y_n + y_m}{2}} \ge d$.

\smallskip
\noindent \textbullet~Подставляя в равенство, получаем $4d^2 + \norm{y_n - y_m}^2 < 2(d + \dfrac{1}{n})^2 + 2(d + \dfrac{1}{m})^2 = 4d^2 + \dfrac{4d}{n} + \dfrac{4d}{m} + 
\dfrac{2}{n^2} + \dfrac{2}{m^2} \Longleftrightarrow \norm{y_n - y_m}^2 < 2(d + \dfrac{1}{n})^2 + 2(d + \dfrac{1}{m})^2 = \dfrac{4d}{n} + \dfrac{4d}{m} + \dfrac{2}{n^2} +
\dfrac{2}{m^2}$. Выражение в правой части $\to 0$ при $n, m \to \infty$, а тогда и $\norm{y_n - y_m} \to 0$. По полноте $H$ $y_n, y_m \to y$. $M$ - замкнутое, $y_n \in M 
\Rightarrow y \in M$.

\noindent \textbullet~$d \le \norm{x - y_n} < d + \dfrac{1}{n}$, $y_n \to y \in M$. Устремляя $n \to \infty$ получаем, что $\norm{x - y} = d = \rho(x, M)$.

\noindent \textbullet~Докажем единственность. От противного: пусть нашлось $y^* \in M : \norm{x - y^*} = d$. Пишем равенство параллелограмма $2 \norm{x - y}^2 + 
2 \norm{x - y^*}^2 = \norm{2x - y - y^*}^2 + \norm{y - y^*}^2$. $\norm{x - y} = d$, $\norm{x - y^*} = d$, $\norm{2x - y - y^*} \ge 4 d^2$. Тогда получаем 
$4d^2 + \norm{y - y^*} \le 4d^2 \Rightarrow \norm{y - y^*} = 0 \Rightarrow y^* = y$. Найденный $y$ есть $\hat{x}$, который мы искали.
\end{proof}

