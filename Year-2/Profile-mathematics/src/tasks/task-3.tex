\subsection*{Невырожденная матрица.}

\noindent \textasteriskcentered~Квадратная матрица называется \textit{невырожденной}, если определитель $\Delta = det A$ не равен нулю: $\Delta = det A \neq 0$. В противном случае матрица называется \textit{вырожденной}.

\subsection*{Союзная матрица.}

\noindent \textasteriskcentered~Матрицей, \textbf{\textit{союзной}} к матрице $A$, называется матрица 
\[
    A^* = 
    \begin{pmatrix}
        A_{11} & A_{21} & \dots & A_{n1} \\
        A_{12} & A_{22} & \dots & A_{n2} \\
        \vdots & \vdots &\ddots & \vdots \\
        A_{1n} & A_{2n} & \dots & A_{nn}
    \end{pmatrix}
\]
, где $A_{ij}$ - алгебраическое дополнение элемента $a_{ij}$ данной матрицы.

\subsection*{Обратная матрица.}

\noindent \textasteriskcentered~Матрица $A^{-1}$ называется \textbf{\textit{обратной}} матрице $A$, если выполнено условие $A \cdot A^{-1} = A^{-1} \cdot A = E$

\bigskip
\noindent \textbf{Теорема}. Всякая невырожденная матрица имеет обратную. ($\exists A^{-1} \Longleftrightarrow detA \neq 0$)

\begin{proof}
\par \noindent \textit{Необходимость}. Пусть $\exists A^{-1} \Rightarrow A \cdot A^{-1} = E$. Так как $det(A \cdot A^{-1}) = det A * det A^{-1} \neq 0 \; (det E = 1 => det A \neq 0)$.

\smallskip
\noindent\textit{Достаточность}. Пусть $det A \neq 0$. Рассмотрим $A \cdot (A^{*})^T$.

\[
   A \cdot A^{*} = 
   \begin{pmatrix}
        A_{11} \cdot a_{11} + A_{12} \cdot a_{12} + \dots + A_{1n} \cdot a_{1n} & 
        0 & \dots & 0 \\ 
        0 & A_{21} \cdot a_{21} + \dots + A_{2n} \cdot a_{2n} &
        \dots & 0 \\ 
        \vdots & \vdots & \ddots & \vdots \\
        0 & 0 & \dots & 
        A_{n1} \cdot a_{n1} + \dots + A_{nn} \cdot a_{nn}  
   \end{pmatrix} = 
\]

$= det A \cdot E \Rightarrow \dfrac{A \cdot (A^*)^T}{det A} = E$.

\end{proof}