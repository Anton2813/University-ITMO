%! TEX root = main.tex

\section{Математическое ожидание.}

\begin{task}{1}
  Найти математическое ожидание величины $\tau$, определенной в задаче 14 гл. 5.
\end{task}

\begin{solution} Полученный результат в задаче 14:
\[
p_\tau(k) = P(\tau = k) = (1 - p)^{k - 1} p
.\] \[
M\tau = \sum_{n=1}^{\infty} n ( 1 - p)^{n - 1} p = p \sum_{n=0}^{\infty} (n + 1) (1 - p)^{n} 
.\] 
Вычислим, что пригодится для нахождения суммы ряда: \[
S_0 = \sum_{n=0}^{\infty} x^n = 1 + x + x^{2} + x^{3} + \dots
.\] 
\[
x S_0 = x \sum_{n=p}^{\infty} x^{n} = x + x^{2} + x^{3} + x^{4} + \dots
.\] 
\[
S_0 - xS_0 = 1 \implies S_0 = \frac{1}{1 - x}
.\] 
Найдем сумму ряда $M\tau$: \[
  (1 - p) M\tau = p \sum_{n=0}^{\infty} (n + 1)(1 - p)^{n + 1}
.\] 
\[
M\tau - (1 - p)M\tau = p \sum_{n=0}^{\infty} (n + 1) (1 - p)^{n} - p \sum_{n=0}^{\infty} (n + 1)(1 -
p)^{n + 1} = p \left(1 + (1 -p) + (1 - p)^{2} + \dots\right) = 1
.\] 
\[
M\tau = \frac{1}{p}
.\] 
\end{solution}

\begin{result}
$\frac{1}{p}$.
\end{result}
%---------------------------------------------------------------------------------------------------
\medskip
\begin{task}{2}
  Обозначим $\xi$ номер испытания, в котором появился нужный ключ (см. пример 3 из пар. 1 гл. 2). Найти
  $M\xi$.
\end{task}

\begin{solution}
Из примера мы знаем, что вероятность того, что на $k$ испытании появился нужный ключ, равна
$\frac{(n- 1)!}{n} = \frac{1}{n}$, так как на $k$ месте лежит требуемый ключ, а остальные $n - 1$
позиции заполняются случайным образом. Тогда \[
M\xi = \sum_{k = 1}^{n} \frac{k}{n} = \frac{1 + 2 + \dots + n}{n} = \frac{\left(1 + n\right) n}{2n}
= \frac{1 + n}{2}
.\] 
\end{solution}

\begin{result}
$\frac{1 + n}{2}$.
\end{result}

%---------------------------------------------------------------------------------------------------
\medskip
\begin{task}{3}
  Решить задачу 2 в случае с возвращением ключей.
\end{task}

\begin{solution}
Данное распределение подчиняется геометрическому: \[
P(\tau = k) = \frac{1}{n} \left(\frac{n - 1}{n}\right)^{k - 1}
.\] Тогда матожидание данной величины: \[
M\tau = \sum_{k = 1}^{\infty} k \frac{1}{n} \left(\frac{n - 1}{n}\right)^{k - 1}  = \frac{1}{n}
\sum_{k = 0}^{\infty} k \left(\frac{n - 1}{n}\right)^{k} = \frac{1}{\left(\frac{1}{n}\right)} = n
\] Последнее равенство получено из задачи 1.
\end{solution}

\begin{result}
$n$.
\end{result}
%---------------------------------------------------------------------------------------------------
\medskip
\begin{task}{6}
  Найти $M\left(\xi_1 + \xi_2\right)$ и $D\left(\xi_1 + \xi_2\right)$, где $\xi_1, \xi_2$ определены
  в задаче 22 гл. 5.
\end{task}

\begin{solution}
\[
M(\xi_1 + \xi_2) = \sum_{i, j = 0}^{9} \left(i + j\right) P(\xi_1 = i, \xi_2 = j) = \sum_{i, j =
0}^{9}\frac{i + j}{100} = \frac{900}{100} = 9
.\] 
\[
D(\xi_1 + \xi_2) = M(\left(\xi_1 + \xi_2\right)^{2}) - \left(M(\xi_1 + \xi_2)\right)^{2} = 97.5 - 81
= 16.5
.\] 
\end{solution}

\begin{result}
$M(\xi) = 9$, $D(\xi) = 16.5$.
\end{result}
%---------------------------------------------------------------------------------------------------
\medskip
\begin{task}{7}
  Пусть $\xi$ --  число комбинаций НУ в $n + 1$ испытаниях схемы Бернулли. Найти $M\tau, D\tau$.
\end{task}

\begin{solution}
  Рассмотрим $\tau$ как сумму $\xi_i$, $P(\xi_i = 1) = p$, $P(\xi_i = 0) = q$: \[
\tau = \xi_1 + \xi_2 + \dots + \xi_n
.\] 
Тогда матожидание данной величины: \[
M(\tau) = M(\xi_1) + M(\xi_2) + \dots + M(\xi_n) = n(p \cdot (1 - p)) = npq
.\] 
\[
D(\tau) ?
.\] 
\end{solution}

\begin{result}
$M(\tau) = npq$, $D(\tau) = ?$
\end{result}

%---------------------------------------------------------------------------------------------------
\medskip
\begin{task}{8}
  Из 100 карточек с числами $00, 01, 02, \dots, 98, 99$ наудачу вынимается одна. Пусть $\eta_1, \eta_2$ 
  -- соответственно сумма и произведение цифр на вынутой карточке. Найти $M\eta_1, D\eta_1, M\eta_2, D\eta_2$.
\end{task}

\begin{solution} В задании 1 было найдено для $\eta_1$:
\[
M(\eta_1) = 9, \; D(\eta_1) = 16.25
.\] 
\[
M(\eta_2) = 20.25, \; D(\eta_2) = M(\eta_2)^2 - (M(\eta_2))^2 = 402.1875
.\] 
\end{solution}

\begin{result}
$M(\eta_1) = 9, \; D(\eta_1) = 16.25$, $M(\eta_2) = 20.25, \; D(\eta_2) = 402.1875$.
\end{result}
%---------------------------------------------------------------------------------------------------
\medskip
\begin{task}{9}
  Для величин $\xi_1, \xi_2$, определенных в задаче 11 гл. 5, найти $M\xi_1, M\xi_2, D\xi_1, D\xi_2,
  cov(\xi_1, \xi_2)$.
\end{task}

\begin{solution}
\[
M(\xi_1) = -1 \cdot \frac{1}{2} + 1 \cdot  \frac{1}{2} = 0
.\] 
\[
M(\xi_2) = -1 \cdot  \frac{1}{3} + 1 \cdot  \frac{5}{12} = \frac{1}{12}
.\] 
\[
D(\xi_1) = M(\xi_1)^2 - (M(\xi_1))^2 = 1 - 0 = 1
.\] 
\[
D(\xi_2) = M(\xi_2)^2 - (M(\xi_2))^2 = \frac{9}{12} - \frac{1}{144} = \frac{107}{144}
.\] 
\[
cov(\xi_1, \xi_2) = M\xi_1\xi_2 - M(\xi_1) \cdot M(\xi_2) = -\frac{1}{4} - 0 = -\frac{1}{4}
.\] 
\end{solution}

\begin{result}
$M(\xi_1) = 0$, $M(\xi_2) = \frac{1}{12}$, $D(\xi_1) = 1$, $D(\xi_2) = \frac{107}{144}$, $cov(\xi_1,
\xi_2) = -\frac{1}{4}$
\end{result}
%---------------------------------------------------------------------------------------------------
\medskip
\begin{task}{10}
  Совместное распределение величин $\xi_1, \xi_2$ определяется формулами $P\left(\xi_1 = 0, \xi_2 = 1\right) =
  P(\xi_1 =0, \xi_2 = -1) = P(\xi_1 = 1, \xi_2 = 0) = P(\xi_1 = -1, \xi_2 = 0) = \frac{1}{4}$.
  Найти $M\xi_1, M\xi_2, D\xi_1, D\xi_2, cov(\xi_1, \xi_2)$. Являются ли $\xi_1, \xi_2$ независимыми
  величинами? 
\end{task}

\begin{solution}
\par\medskip
\begin{tabular}{|c|c|c|c|}
  \hline
  $\xi_2  \setminus \xi_1$ & -1 & 0 & 1 \\
  \hline
  -1 & 0 & 1/4 & 0 \\
  \hline
  0 & 1/4 & 0 & 1/4 \\
  \hline
  1 & 0 & 1/4 & 0 \\
  \hline
\end{tabular}
\[
  M(\xi_1) = -\frac{1}{4} + \frac{1}{4} = 0
.\] 
\[
  M(\xi_2) = -\frac{1}{4} + \frac{1}{4} = 0
.\] 
\[
D(\xi_1) = \frac{1}{4} + \frac{1}{4} = \frac{1}{2}
.\] 
\[
D(\xi_2) = \frac{1}{4} + \frac{1}{4} = \frac{1}{2}
.\] 
\[
cov(\xi_1, \xi_2) = 0
.\] 

\end{solution}

\begin{result}
$M(\xi_1) = 0$, $M(\xi_2) = 0$ , $D(\xi_1) = \frac{1}{2}$, $D(\xi_2) = \frac{1}{2}$, величины
зависимы. 
\end{result}

%---------------------------------------------------------------------------------------------------
\medskip
\begin{task}{11}
  Случайные величины $\xi_1, \xi_2,\xi_3, \xi_4,\xi_5$ независимы; $D\xi_i = \sigma^2$. Найти 
  коэффициент корелляции величин а) $\xi_1 + \xi_2, \; \xi_3+\xi_4+\xi_5$ б) $\xi_1 +\xi_2+\xi_3, \;
  \xi_3 +\xi_4+\xi_5$. 
\end{task}

\begin{subtask}{а)}
  Пусть $\eta_1 = \xi_1 + \xi_2$, $\eta_2 = \xi_3 + \xi_4 + \xi_5$. Тогда \[
  D(\eta_1 + \eta_2) = D(\eta_1) + D(\eta_2) + 2 cov(\eta_1, \eta_2)
  .\] 
  \[
  D(\eta_1 + \eta_2) = D(\xi_1 + \xi_2 + \xi_3 + \xi_4 + \xi_5) = 5 \sigma^2
  .\] 
  \[
  D(\eta_1) = 2 \sigma^2, \; D(\eta_2) = 3 \sigma^2
  .\] 
  \[
  cov(\eta_1, \eta_2) = \frac{1}{2} \left(D(\eta_1 + \eta_2) - D(\eta_1) - D(\eta_2)\right) = 0
  \implies \rho(\eta_1, \eta_2) = 0
  .\] 
\end{subtask}

\begin{subtask}{б)}
Пусть $\eta_1 = \xi_1 + \xi_2 + \xi_3$, $\eta_2 = \xi_3 + \xi_4 + \xi_5$.
\[
D(\eta_1 + \eta_2) = D(\eta_1) + D(\eta_2) + 2 cov(\eta_1, \eta_2)
.\] 
\[
D(\eta_1 + \eta_2) = D(\xi_1) + D(\xi_2) + 4 D(\xi_3) + D(\xi_4) + D(\xi_5) = 8 \sigma^2
.\] 
\[
D(\eta_1) = 3 \sigma^2
.\] 
\[
D(\eta_2) = 3 \sigma^2
.\] 
\[
cov(\eta_1, \eta_2) = \frac{1}{2} 2 \sigma^2 = \sigma^2
.\] 
\[
\rho(\eta_1, \eta_2) = \frac{cov(\eta_1, \eta_2)}{\sqrt{D(\eta_1) D(\eta_2)} } =
\frac{\sigma^2}{3\sigma^2} = \frac{1}{3}
.\] 
\end{subtask}

\begin{result}
а) $0$, б) $\frac{1}{3}$.
\end{result}
%---------------------------------------------------------------------------------------------------
\medskip
\begin{task}{16}
  По $n$ конвертам случайно разложили $n$ писем различным адресатам. Найти вероятность того, что
  хотя бы одно письмо попадет своему адресату. Найти предел этой вероятности при $n \to \infty$.
\end{task}

\begin{solution}
Задача на включения-исключения. Сначала найдем вероятность, что хотя бы одно письмо дошло. Если не
исключать пересечения событий, то получаем, что для каждого $k$ письма вероятность события $k$
письмо дошло до адресата равна $(n - 1)!$, так как мы фиксируем одно письмо, а все остальные
раскладываем как хотим. В сумме $n \cdot (n - 1)! = n!$. Теперь исключим такие события, например как
1-ое письмо дошло до адресата и 2-ое письмо дошло до адресата, так как первоначально для $k = 1$ и 
$k = 2$ точно входит этот дубликат.

Когда мы выбирали множество, в которое точно входит один необходимый элемент, то количество способов
выбрать такое множество было равно $n$. Теперь когда необходимо выбрать как минимум $2$ элемента,
при этом повторяющиеся события $P(AB), P(BA)$, то
количество способов выбрать такое подмножество из $n$ элементов равно $C_n^2$.

И так далее включая-исключая пересечения-удаления получим, что кол-во способов выбрать события, что
хотя бы одному дошло письмо, равняется: 
\[
P(A) = n \cdot \frac{1}{n} - C_n^2 \frac{1}{n(n-1)} + C_n^3 \frac{1}{n(n - 1)(n -2)} - \dots +
(-1)^{n - 1} \frac{1}{n!} = 1 - \frac{1}{2!} + \frac{1}{3!} - \dots + (-1)^{n - 1}\frac{1}{n!} 
.\] 
По формуле Тейлора:
\[
  P(A) \to  1 - e^{-1} \approx 0.63212
.\] 
\end{solution}

\begin{result}
$0.63212 $.
\end{result}


%---------------------------------------------------------------------------------------------------
\medskip
\begin{task}{17}
В задаче 16 найти математическое ожидание и дисперсию числа $\xi$ писем, попавших своему адресату.
\end{task}

\begin{solution}
Воспользуемся индикаторами: \[
\xi = \xi_1 + \xi_2 + \xi_3 + \dots + \xi_n
.\] 
В $i$ конверт требуемое письмо попадает в среднем с вероятностью $\frac{1}{n}$. Тогда математическое ожидание: \[
M(\xi) = M(\xi_1) + M(\xi_2) + \dots + M(\xi_n) = n \cdot \frac{1}{n} = 1 
.\] 
Так как $\xi_k^2(w) = \xi_k(w)$, то \[
\xi^2 = \sum_{i=1}^{\infty} \xi^2_i + \sum_{i \neq j} \xi_i \xi_j = \sum_{i = 1}^{\infty} \xi_i +
\sum_{i \neq j}\xi_i \xi_j = \xi + \sum_{i \neq  j}\xi_i \xi_j
.\] 
Таким образом, \[
  D(\xi) = M\xi^2 - (M\xi)^2 = \sum_{i \neq j} M \xi_i \xi_j + M \xi - (M\xi)^2 = \frac{n \cdot (n -
  1)}{n \cdot (n - 1)} + 1 - 1 = 1
.\] 
\end{solution}

\begin{result}
$M(\xi) = 1$, $D(\xi) = 1$.
\end{result}

