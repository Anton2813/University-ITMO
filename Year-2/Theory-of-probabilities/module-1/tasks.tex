%! TEX root = main.tex

\section{Вероятностное пространство.}
% \subimport{tasks}{task-1.tex}

%---------------------------------------------------------------------------------------------------
\begin{task}{1} 
  Проверить следующие соотношения между событиями: 
\end{task}


\begin{subtask}{1}
  $A \setminus B = A \overline{B}$
\end{subtask}

\begin{solution}
Пусть $\omega \in A \setminus B$. Тогда $\omega \in A$, $\omega \not\in B$, $
\omega \in \overline{B}$. Это означает, что $\omega \in A\overline{B}$. 
В обратную сторону, $\omega \in A\overline{B} \Rightarrow \omega \in A$, $\omega \in 
\overline{B}$, $\omega \not\in B \Rightarrow \omega \in A \setminus B$.
\end{solution}


\begin{subtask}{2}
$A \setminus B = A \setminus AB = \left( A + B \right)  \setminus B$.
\end{subtask}

\begin{solution}
a) Пусть $\omega \in A \setminus B \Rightarrow \omega \in A$, $\omega \not\in AB$, так 
как $\omega \not\in B \Rightarrow \omega \in A \setminus AB$. 
b) Пусть $\omega \in A, \omega \not\in
AB$. Из второго $\omega \not\in$ или $A$ или $B$, но поскольку $\omega \in A$, то $\omega \not\in
B$. А тогда получаем $\omega \in A \setminus B$, а это тоже самое, что и $\omega \in (A + B) \setminus
B$, так как $(A + B) \setminus B = (A \setminus B) + (B \setminus B) = A \setminus B + \emptyset = 
A \setminus B$. 
c) Пусть $\omega \in (A + B) \setminus B$, по уже проделанному выше получаем $\omega
\in A \setminus B$.
\end{solution}


\begin{subtask}{3}
$\overline{\left( A + B \right) } = \overline{A}~\overline{B}$, $\overline{AB} = \overline{A} + 
\overline{B}$
\end{subtask}

\begin{solution}
a) Пусть $\omega \in \overline{A + B} \Rightarrow \omega \not\in A$, $\omega \not\in B 
\Leftrightarrow \omega \in \overline{A}, \omega \in \overline{B} \Rightarrow \omega \in \overline{A}~
\overline{B}$. Обратное доказывается аналогично. b) Пусть $\omega \in \overline{AB} \Leftrightarrow 
\omega \not\in AB$. Тогда $\omega \not\in A$ или $\omega \not\in B$ или в обоих сразу. НУО считаем, что
$\omega \not\in A \Rightarrow \omega \in \overline{A}$, а значит что $\omega \in \overline{A} + 
\overline{B}$. Для остальных случаев доказывается аналогично. В обратную сторону, пусть $\omega \in
\overline{A} + \overline{B}$. НУО считаем, что $\omega \in \overline{A} \Rightarrow \omega \not\in A$, 
а тогда $\omega \not\in A B \Leftrightarrow \omega \in \overline{AB}$.
\end{solution}


\begin{subtask}{4}
$A \left( B \setminus C \right) = AB \setminus AC$
\end{subtask}

\begin{solution}
a) Пусть $\omega \in A$, $\omega \in \left( B \setminus C \right) \Leftrightarrow \omega
\in A $, $\omega \in \left( B \setminus C \right) $. Из второго $\omega \in B, \omega \in \overline{C}$.
Тогда $\omega \in AB$ и $\omega \in A\overline{C} \Rightarrow \omega \not\in AC$, поскольку
$\omega$ должен лежать в $A$ и $C$. Получаем $\omega \in AB \overline{\left( AC \right)} $, а это по первому пункту 
есть $\omega \in AB \setminus AC$. b) Обратно по первому пункту получаем $\omega \in AB \overline{\left( AC \right)}
 \Rightarrow \omega \in AB$, $\omega \not\in AC$. Поскольку $\omega \in AB \Rightarrow \omega \in A, 
 \omega \in B$ и $\omega \not\in AC \Rightarrow \omega \not\in C \Rightarrow \omega \in A, \omega
\in B\overline{C} \Rightarrow \omega \in A \left( B \setminus C \right)$. 
\end{solution}


%---------------------------------------------------------------------------------------------------
\medskip
\begin{task}{2}
  Установить, какие из следующих соотношений правильные:
\end{task}


\begin{subtask}{1}
  $A \setminus \left( B \setminus C \right) = \left( A \setminus B \right) + C$
\end{subtask}

\begin{solution}
С левой стороны $A \setminus \left( B \setminus C \right) = A \overline{\left( 
B  \overline{C}\right) } = A \left( \overline{B} + C \right) = A \setminus B + A C$. Неверно.
\end{solution}


\begin{subtask}{2}
$A \setminus \left( B \setminus C \right) = \left( A \setminus B \right) + AC$
\end{subtask}

\begin{solution}
Верно и доказано в пункте 1.
\end{solution}


\begin{subtask}{3}
 $\left( A + B \right) \setminus C = \left( A \setminus C \right) + \left( B \setminus C \right) $ 
\end{subtask}

\begin{solution}
$\left( A + B \right) \setminus C \Leftrightarrow \omega \in \left(A + B\right),
\omega \in \overline{C} \Leftrightarrow \omega \in A\overline{C} $ или $ \omega \in B\overline{C}$, 
а это тоже самое что и $\omega \in \left(A \setminus C\right) + \left(B \setminus C\right)$. Верно
\end{solution}


\begin{subtask}{4}
$\left( A + B \right)  \setminus C = A + \left( B \setminus C \right) $
\end{subtask}

\begin{solution}
Неверно, доказано обратное в пункте выше.
\end{solution}


\begin{subtask}{5}
$A\overline{B}C \subset A + B$
\end{subtask}

\begin{solution}
Множество, получающееся справа как минимум не меньше, чем $A$, а значит оно является 
подмножеством $A + B$.
\end{solution}


\begin{subtask}{6}
$\left(A \setminus B\right)\left(C \setminus D\right) = AC \setminus BD$
\end{subtask}

\begin{solution}
$\left(A \setminus B\right) \left(C \setminus D\right) = A\overline{B}C\overline{D} =
\left(AC\right)\left(\overline{B}~\overline{D}\right) = \left(AC\right)\overline{\left(
B + D\right)} = \left(AC\right) \setminus \left(B + D\right)$.

\noindentНеверно.
\end{solution}


\medskip
\begin{result}
неверно: 1), 4), 6); верно: 2), 3), 5)
\end{result}


%---------------------------------------------------------------------------------------------------
\medskip
\begin{task}{3}
Упростить следующие выражения:
\end{task}

\begin{solution}

\smallskip
\noindent~1) $A + AB = A$ 

\medskip
\noindent~2) $\left(A + B\right)\left(A + \overline{B}\right) = A \left(B + \overline{B}\right) = A$

\medskip
\noindent~3) $\left(A \setminus C\right)\left(B \setminus \overline{C}\right) = \left(A\overline{C}\right)
\left(BC\right) = A\overline{C}BC = \emptyset$

\medskip
\noindent~4) $\left(A + B\right)\left(\overline{A} + B\right)\left(A + \overline{B}\right) = 
\left(A + \overline{A}\right)B\left(A + \overline{B}\right) = B\left(A + \overline{B}\right) = 
BA + B\overline{B} = BA$
\end{solution}


\begin{result}
1) $A$, 2) $A$, 3) $\emptyset$, 4) $BA$
\end{result}


%---------------------------------------------------------------------------------------------------
\medskip
\begin{task}{4}
Пусть
\[
  A_n = \left\{x: a \le x < a + \frac{1}{n}\right\} 
.\] 
\[
  B_n = \left\{x: a \le x \le b - \frac{1}{n}\right\} 
.\] 
Для событий найти более простые выражения.
\[
  A = \bigcap_{n = 1}^{\infty} A_n;~~~ B = \bigcup_{n = 1}^\infty B_n
.\] 
\end{task}

\begin{solution}

\smallskip
\noindent a) При $n \to \infty$ $x$ для $A_n$ стремится к $a$, поскольку он ограничен сверху
выражением, которое стремится к $a$ ($\lim_{n \to \infty} a + \frac{1}{n} = a$). Поэтому $A = a$, если мы рассматриваем всюду плотное множество.

\medskip
\noindent b) При $n \to \infty$ для $B_n$ $x : a \le x \le b$, поскольку $\lim_{n \to \infty} b - \frac{1}{n} = b $.
Поэтому $B$ = $\left\{x: a \le x \le b\right\}$.
\end{solution}

\begin{result}
a) $A = a$, b) $B = \left\{x: a \le x \le b\right\}$
\end{result}


%---------------------------------------------------------------------------------------------------
\medskip
\begin{task}{5}
Какие подмножества множества $\Omega$ в примере 3 из пар.1 при $n = 3$ соответствуют событиям:
\end{task}

\begin{solution}
\par\smallskip
\noindent~1) При первом подбрасывании выпал герб: $A = \{$ГРР, ГРГ, ГГР, ГГГ$\}$

\medskip
\noindent~2) Всего выпало ровно 2 герба: $A = \{$РГГ, ГРГ, ГГР$\}$

\medskip
\noindent~3) Выпало не более одного герба: $A = \{$РРР, РРГ, РГР, ГРР$\}$
\end{solution}

\begin{result}
  1) $\{$ГРР, ГРГ, ГГР, ГГГ$\}$, 2) $\{$РГГ, ГРГ, ГГР$\}$, 3) $\{$РРР, РРГ, РГР, ГРР$\}$
\end{result}


%---------------------------------------------------------------------------------------------------
\medskip
\begin{task}{6}
Пусть $A, B, C$ -- три произвольных события. Найти выражения для событий, состоящих
в том, что из $A, B, C$:
\end{task}

\begin{solution}

\smallskip
\noindent~1) $A\overline{B}\overline{C}$

\smallskip
\noindent~2) $AB\overline{C}$

\smallskip
\noindent~3) $ABC$ 

\smallskip
\noindent~4) $\Omega - \overline{A}\overline{B}\overline{C} = A + B + C$

\smallskip
\noindent~5) $A\overline{B}\overline{C} + \overline{A}B\overline{C} + \overline{A}\overline{B}C$

\smallskip
\noindent~6) $\overline{A}\overline{B}\overline{C}$

\smallskip
\noindent~7) $\Omega - ABC = \overline{A} + \overline{B} + \overline{C}$
\end{solution}


%---------------------------------------------------------------------------------------------------
\medskip
\begin{task}{7}
Пусть в примере 3 из пар.1 $n = 3$. Является ли алгеброй следующая система подмножеств:
\end{task}

\begin{solution}

\smallskip
\noindent~$\emptyset, \Omega, \{$ГГГ, ГРГ, ГГР, ГРР$\}, \{$РГГ, РРГ, РГР, РРР$\}$

\smallskip
\noindent $\emptyset, \Omega$ входят точно, так как если $\Omega \in U$, то $\emptyset = \Omega \setminus
\Omega \in U$, также если $\{$ГГГ, ГРГ, ГГР, ГРР$\} \in U$, то и $\Omega \setminus \{$ГГГ, ГРГ, ГГР, ГРР$\} = 
\{$РГГ, РРГ, РГР, РРР$\} \in U$. Верно.
\end{solution}

\begin{result}
Да.
\end{result}


%---------------------------------------------------------------------------------------------------

