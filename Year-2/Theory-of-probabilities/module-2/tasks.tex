%! TEX root = main.tex
\section{Простейшие вероятностные схемы и их обобщения}

%---------------------------------------------------------------------------------------------------
\begin{task}{1}
Брошено 2 игральные кости. Предполагается, что элементарные события равновероятны,
найти вероятность события, что:
\end{task}

\begin{solution}
\par\smallskip
\noindent~1) $A = \{$на первой кости выпала 1 $\} = \frac{1}{6}$ 

\medskip
\noindent~2) $\overline{A} = \frac{5}{6}$ 

\medskip
\noindent~3) $B = \{$выпала хотя бы одна 6$\} = \frac{11}{36}$ - всего возможных исходов - $36$, из 
которых удовлятворяющих условию $11 = (1,6), (2,6), (3,6), (4,6), (5,6), (6,6), (6,5), (6,4),
(6,3), (6,2), (6,1)$.

\medskip
\noindent~4) $A \overline{B}$ - на первой кости выпала 1 и не выпала 6. Результат: $\frac{1}{6} \cdot  
\frac{5}{6} = \frac{5}{36}$ . 
\end{solution}

\begin{result}
1) $\frac{1}{6}$, 2) $\frac{5}{6}$, 3) $\frac{5}{36}$
\end{result}


%---------------------------------------------------------------------------------------------------
\medskip
\begin{task}{2}
Очередная задача про книжные полки.
\end{task}

\begin{solution}
Давайте вместо того, чтобы считать все вероятности книг, а оттуда выбирать количество, 
нам удовлетворяющих, возьмем 2 книги, которые нам нужны и скрутим их скотчем. Теперь у нас не 
$n$ книг, а $n - 1$ книга, кроме того, мы можем взять сначала первую книгу со второй, а можем 
и наоборот, поэтому итоговая вероятность: $\frac{(n-1)! \cdot 2}{n!} = \frac{2}{n}$.
\end{solution}

\begin{result}
$\frac{2}{n}$
\end{result}


%---------------------------------------------------------------------------------------------------
\medskip
\begin{task}{3}
Числа 1,2,\ldots,n расставлены случайным образом. Предполагая, что различные 
расположения чисел равновероятны, найти вероятность того, что числа 1,2,3 расположены в порядке
возрастания, но не обязательно рядом.
\end{task}

\begin{solution}
$P(\Omega) = 1$, при расставлении чисел от $1,..,n$ случайным образом возможны такие 
конфигурации: $123$, $132$, $312$, $321$, $213$, $231$, где между цифрами или до, или после может что-то стоять,
а может и не стоять. Так как каждое из этих событий равновероятно, а нам подходит лишь 1 событие
из 6, то искомая вероятность равна $\frac{1}{6}$.
\end{solution}

\begin{result}
$\frac{1}{6}$
\end{result}


%---------------------------------------------------------------------------------------------------
\medskip
\begin{task}{6}
Сравнить вероятности событий:
\end{task}


\begin{subtask}{1}
$A = \{$при одновременном бросании четырех костей выпала хотя бы одна 1$\}$.
\end{subtask}

\begin{solution}
\noindentВероятность того, что на всех костях не выпадет 1 :$\frac{5}{6} \cdot \frac{5}{6} \cdot \frac{5}{6} \cdot
\frac{5}{6}$. Значит вероятность того, что хотя бы на одном выпадет: $1 - \frac{5^4}{6^4} \approx 0.51774691$
\end{solution}


\begin{subtask}{2}
$B = \{$При 24 бросаниях двух костей выпали хотя бы 1 раз две 1$\}$.
\end{subtask}

\begin{solution}
Аналогично тому, что выше: вероятность, что не выпадет: $\left(\frac{35}{36}\right)^{24}$,
значит искомая вероятность: $1 - \left(\frac{35}{36}\right)^{24} \approx 0.491403876$. Результат:
вероятность 1) больше.
\end{solution}


\medskip
\begin{result}
первая вероятность больше.
\end{result}


%---------------------------------------------------------------------------------------------------
\medskip
\begin{task}{7}
В чулане $n$ пар ботинок. Из них случайно выбирается $2r$ ботинок. Найти
вероятность того, что среди выбранных ботинок.
\end{task}

\begin{subtask}{a)}
Нет парных. Всего вероятность выбрать $2r$ ботинок из $n$ пар - $C_{2n}^{2r}$. 
Количество событий, которые удовлетворяют: необходимо выбрать $2r$ ботинок из $n$, чтобы
каждый из них был либо левым, либо правым ($2r < n$), что составляет $C_n^{2r}$. Кроме того, 
ботинок может быть либо левым, либо правые, то есть возможно для $2r$ ботинок $2^{2r}$ возможных
вариантов. Таким образом, результат:  $\frac{4^r\cdot C_{n}^{2r}}{C_{2n}^{2r}}$.
\end{subtask}

\begin{subtask}{b)}
Одна пара. Знаменатель остается тем же, в числителе будет $2^{2r - 2} \cdot C_{n - 1}^{2r - 2}$ - вариантов
выбрать $n - 1$ одиночный сапог, умноженное на $n$ - вариантов выбрать полноценную пару. Результат: 
$\frac{n 2^{2r - 2}C_{n- 1}^{2r - 2}}{C_{2n}^{2r}}$.
\end{subtask}

\medskip
\begin{result}
a)$\frac{4^r\cdot C_{n}^{2r}}{C_{2n}^{2r}}$, b) $\frac{n 2^{2r - 2}C_{n- 1}^{2r - 2}}{C_{2n}^{2r}}$.
\end{result}


%---------------------------------------------------------------------------------------------------
\medskip
\begin{task}{8}
В партии изделий 90 исправных и 10 бракованных. Найти вероятность того, что среди
10 проданных изделий.
\end{task}


\begin{subtask}{a)}
Ровно одно бракованное.
\end{subtask}

\begin{solution}
10 изделий из 100 можно взять $C_{100}^{10}$ способами. Теперь количество вариантов, 
которые удовлетворяют условию равно $C_{10}^1$ - выбираем 1 бракованное изделие из 10, $C_{90}^{9}$ 
- выбираем 9 исправных изделий из 90 всех исправных. Итого: $\frac{10 \cdot C_{90}^{9}}{C_{100}^{10}}
\approx 0.40799532$.
\end{solution}


\begin{subtask}{b)}
Нет бракованных.
\end{subtask}

\begin{solution}
Рассуждаем аналогично, только в числителе теперь $C_{90}^{10}$ - кол-во способов выбрать
10 небракованных изделий. Результат: $\frac{C_{90}^{10}}{C_{100}^{10}} \approx 0.330476211$.
\end{solution}


\medskip
\begin{result}
a) $0.40799532$, b) $0.330476211$
\end{result}


%---------------------------------------------------------------------------------------------------
\medskip
\begin{task}{11}
Из множества чисел по схеме выбора с возвращением найти вероятность попадания в
круг радиуса n
\end{task}

\begin{solution}
\noindent Рассмотрим круг радиуса $n$, его площадь равна $\pi * n^2$. Площадь квадрата, в которой 
располагается этот круг равна $4 * n^2$, а значит отношение площадей, которое занимает круг в 
квадрате, составляет $\frac{\pi * n^2}{4 * n^2} = \frac{\pi}{4}$. Это и есть ответ, поскольку
выбор координат не имеет приоритетов (равновероятный).
\end{solution}

\begin{result}
$\frac{\pi}{4}$
\end{result}


%---------------------------------------------------------------------------------------------------
\medskip
\begin{task}{13}
Найти вероятность размещения частиц по ячейкам
\end{task}

\begin{solution}
\par\noindent~a) Вероятность того, что займутся все ячейки равна $\frac{n!}{n^n}$, где в знаменателе 
сколько всего различных вариантов, а в числителе - количество вариантов, которые нам подходят (сначала
кладем частицу в какую-то из $n$ ячеек, затем в какую-то из $n - 1$, и так далее). Значит вероятность
того, что в какой-то из ячеек не будет частицы равна $1 - \frac{n!}{n^n}$.

\medskip
\noindent~b) Всего количество событий также равняется $n^n$. Сначала разложим $n - 1$ частицу 
по $n$ ячейкам, количество вариантов это сделать - $n!$, теперь положим в какую-то из $n - 1$ 
заполненных ячеек вторую частицу, вероятность это сделать: $n - 1$. Поскольку нам не важно, сначала
мы положили i частицу в ячейку k, а потом j частицу или наоборот, j частицу сначала, а потом i, 
то следует разделить количество этих вариантов на 2. Кроме того, нам 
еще не важно, под каким номером эта частица, она не обязательно последняя, 
значит еще $n$ вариантов. Тогда итоговое количество вариантов:
$n! \cdot (n - 1) / 2$. Результат: $\frac{n! \cdot n \cdot (n - 1) / 2}{n^n}$.
\end{solution}

\begin{result}
$\frac{n! \cdot n \cdot (n - 1) / 2}{n^n}$
\end{result}


%---------------------------------------------------------------------------------------------------
\medskip
\begin{task}{14}
Найти вероятность того, что на две карточки <<Спортлото>> с отмеченными номерами
(4,12,38,20,41,46) и (4,12,38,20,41,49) будет получено по одному минимальному выигрышу (угадано по 3
числа)
\end{task}

\begin{solution}
Общее количество возможных исходов: $C_{49}^6$. Случаи, которые нам удовлетворяют: 
из первых 5 чисел в лото было угадано ровно 3, поскольку они совпадают на обеих карточках, из 
первых 5 чисел в лото было угадано ровно 2, а также на каждой карточке было угадано последнее число.
Рассчитаем количество исходов: $C_5^3 \cdot  C_{42}^3 + C_5^2 \cdot C_{42}^2$, где первый множитель
в каждом слагаемом - количество вариантов, полученных из угаданных цифр, второй - количество вариантов
из неугаданных. В результате получим: $\frac{C_5^3 \cdot  C_{42}^3 + C_5^2 \cdot C_{42}^2}{C_{49}^6} \approx 
0.008825201$
\end{solution}

\begin{result}
$\approx 0.008825201$
\end{result}


%---------------------------------------------------------------------------------------------------
\medskip
\begin{task}{15}
На отрезок $\left[a, b\right]$ наудачу брошена точка
\end{task}

\begin{solution}
Так как сама $F$ является линейной, то ее производная есть константа. В данном случае, 
так как мы рассматриваем отрезок $\left[a, b\right] = \left[0, 1\right] $, то $F'(x) = 1$ на отрезке 
$\left[a, b\right] $.
\end{solution}

\begin{result}
$F'(x) = 1, x \in \left[a, b\right]$
\end{result}
%---------------------------------------------------------------------------------------------------

