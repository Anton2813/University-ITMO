%! TEX root = main.tex

\section{Последовательности испытаний.}

\begin{task}{1}
Два игрока поочередно извлекают шары (без возвращения) из урны, содержащей 2 белых и 4 черных шара. 
Выигрывает тот, кто первым вынет белый шар. Найти вероятность выигрыша участника, начавшего игру.
\end{task}

\begin{solution}
Для победы первого участника необходимо, чтобы белый шар был вынут им. Для этого белый шар следует
достать на 1, на 3 или на 5 ходу. Посчитаем сумму вероятностей:
\begin{multline*}
  P(A) = P_1(A) + P_3(A) + P_5(A) = P(W) + P(BBW) + P(BBBBW) = 
  \frac{2}{6} + \frac{4}{6} \cdot \frac{3}{5} \cdot 
  \frac{2}{4} + \\ \frac{4}{6} \cdot  \frac{3}{5} \cdot \frac{2}{4} \cdot  \frac{1}{3} \cdot  \frac{2}{2} = 
  \frac{2}{6} + \frac{24}{120} + \frac{24}{360} = \frac{120 + 72 + 24}{360} = \frac{3}{5}
\end{multline*}
\end{solution}

\begin{result}
$\frac{3}{5}$.
\end{result}

%---------------------------------------------------------------------------------------------------
\medskip
\begin{task}{2}
Два игрока поочередно извлекают шары (без возвращения) из урны, содержащей 2 белых, 4 черных и 1
красный. Выигрывает тот, кто первым вынет белый шар. Если появляется красный шар, то объявляется ничья.
Пусть $A_1 = \{$ выигрывает игрок, начавший игру $\}$, $A_2 = \{$ выигрывает второй участник$\}$, 
$B = \{$игра закончится вничью$\}$. Найти  $P(A_1), P(A_2), P(B)$.
\end{task}

\begin{solution}
По аналогии с задачей 1:
\begin{flalign*}
P(A_1) = \frac{2}{7} + \frac{4}{7} \cdot \frac{3}{6} \cdot  \frac{2}{5} + \frac{4}{7} \cdot \frac{3}{6}
  \cdot \frac{2}{5} \cdot \frac{1}{4} \cdot \frac{2}{3} = \frac{2}{7} + \frac{24}{210} + \frac{48}{2520} = 
  \frac{720 + 288 + 48}{2520} = \frac{1056}{2520} = \frac{44}{105} &&
\end{flalign*}
\begin{flalign*}
  P(A_2) = \frac{4}{7} \cdot \frac{2}{6} + \frac{4}{7} \cdot \frac{3}{6} \cdot \frac{2}{5} \cdot \frac{2}{4} =
  \frac{8}{42} + \frac{48}{840} = \frac{160 + 48}{840} = \frac{28}{105} &&
\end{flalign*}
\begin{flalign*}
P(B) = \frac{1}{7} + \frac{4}{7} \cdot \frac{1}{6} + \frac{4}{7} \cdot \frac{3}{6} \cdot \frac{1}{5} +
\frac{4}{7} \cdot \frac{3}{6} \cdot \frac{2}{5} \cdot \frac{1}{4} + \frac{4}{7} \cdot \frac{3}{6} \cdot 
\frac{2}{5} \cdot \frac{1}{4} \cdot \frac{1}{3} =  \frac{1}{7} + \frac{4}{42} + \frac{12}{210} + 
\frac{24}{840} + \frac{24}{2520} =  \\
= \frac{840}{2520} = \frac{1}{3} &&
\end{flalign*}
\end{solution}

\begin{result}
$P(A_1) = \frac{44}{105}$, $P(A_2) = \frac{28}{105}$, $P(B) = \frac{1}{3}$.
\end{result}

%---------------------------------------------------------------------------------------------------
\medskip
\begin{task}{3}
Из урны, содержащей $M$ белых и $N - M$ черных шаров, по одному без возвращения извлекают все шары.
Найти вероятности событий:
\[
  A_k = \left\{\text{$k$-й шар белый}\right\}, \;
  B_{k,l} = \left\{\text{$k$-й и $l$-й шары белые}\right\},  \;
  C_{k,l} = \left\{\text{$k$-й шар черный, а $l$-й - белый}\right\} 
.\] 
\end{task}

\begin{solution}
Вероятность того, что случайно взятый шар является белым (шары равномерно распределены, т.е $M$ позиций
белые, $N - M$ позиций черные) равна $\frac{M}{N} = A_k$.
\par\medskip
\noindentСобытию $B_{k,l}$ удовлетворяет последовательность исходов: сначала возьмем из $N$ шаров белый, затем
возьмем из оставшихся шаров еще один белый (порядок выбора не важен). Таким образом, вероятность данного события $\frac{M}{N} \cdot 
\frac{M - 1}{N - 1} = B_{k,l}$.
\par\medskip
\noindentСобытию $C_{k,l}$ удовлетворяет последовательность исходов: сначала возьмем $k$-й шар, при этом
он черный, затем возьмем из оставшихся $N - 1$ шаров $l$-й шар, он должен быть белым. Вероятность события 
$\frac{N - M}{N} \cdot \frac{M}{N - 1} = C_{k,l}$
\end{solution}

\begin{result}
$A_k = \frac{M}{N}$, $B_{k,l} = \frac{M \cdot \left(M - 1\right) }{N \cdot (N - 1)}$, $C_{k,l} 
  = \frac{\left(N - M\right) \cdot M}{N \cdot \left(N - 1\right) }$
\end{result}

%---------------------------------------------------------------------------------------------------
\medskip
\begin{task}{4}
Проведено 10 независимых испытаний, каждое из которых заключается в одновременном подбрасывании трех 
игральных костей. Найти вероятность того, что в четырех испытаниях появятся в точности по две <<6>>.
\end{task}

\begin{solution}
Вероятность события в $i$ испытании выпало 2 <<6>> из 3 кубиков равно 
$C_3^2 \left(\frac{1}{6}\right)^2 \cdot \left(\frac{5}{6}\right)^1 = 3 \cdot \frac{5}{216} = \frac{15}{216}$.
\medskip
\noindentВоспользуемся схемой Бернулли: $P(m = 4) = C_{10}^4 \left(\frac{15}{216}\right)^4 \left(\frac{201}{216}\right)^6 \approx 0.0031712$
\end{solution}

\begin{result}
$\approx 0.0031712$.
\end{result}

%---------------------------------------------------------------------------------------------------
\medskip
\begin{task}{6}
Сколько нужно взять случайных чисел, чтобы число <<6>> появилось хотя бы один раз с вероятностью,
не меньшей a) $0.7$ b) $0.9$?
\end{task}

\begin{solution}
Вероятность события <<на кубике выпала 1 шестерка>> равна $\frac{1}{10}$. Тогда по распределению Бернулли
вероятность того, что шестерка выпала на $k$-м шаге равна  $\frac{9}{10}^{(k - 1)} \cdot \frac{1}{10}$, а
вероятность, что шестерка выпала до или на $k$-м шаге: $\sum_{i = 0}^k \frac{9}{10}^{(i - 1)} \cdot \frac{1}{10}$,
что является геометрической прогрессией. Тогда необходимо найти такое $k$, при котором вероятность
события не меньше 0.7 и 0.9.

\medskip
\noindentПо формуле геометрической прогрессии:
\[
  \frac{\frac{1}{6} \cdot \left(\left(\frac{9}{10}\right)^k - 1\right)}{\frac{9}{10} - 1} = 
  \left(1 - \left(\frac{9}{10}\right)^k\right) \ge  a
.\], где $a = {0.7, 0.9}$.
\smallskip
\noindentРешим данное уравнение при заданных $a$ и найдем, что при $a = 0.7$ $k \ge 11.4272$, $a = 0.9$ $k \ge 21.8543$.
. Ответ округляем до ближайших целых.
\end{solution}

\begin{result}
  $a = 0.7, k = 12$, $a = 0.9, k = 22$.
\end{result}

%---------------------------------------------------------------------------------------------------
\medskip
\begin{task}{8}
Среди $5M$ билетов $M$ выйгрышных. Найти вероятность $Q(n)$ того, что среди $n$ купленных билетов 
есть хотя бы один выигрышный. Вычислить $Q(n)$ при 1) $M = 3$ ; 2) $M = 10$ для $n$, определенных
в задаче 7 в случаях а) 0.65, б) 0.9, в) 0.99.
\end{task}

\begin{solution}
Воспользуемся гиперболическим распределением. Вероятность случая, когда из $5M$ билетов выбирается
$n$ и при этом ни один из них не выйграл: $P_n(A) = \frac{C_M^0 \cdot C_{4M}^n}{C_{5M}^n}$.
\medskip
\noindentТогда вероятность того, что хотя бы один билет выйграл: 
\[
  Q(n) = 1 - P_n(A) = 1 - \frac{C_M^0 \cdot C_{4M}^n}{C_{5M}^n}
.\] 
\end{solution}

\begin{subtask}{1}
Подставим для a), б), в) при $M = 3$:

\medskip
\noindent\textbullet~a) $n = 5$, $Q(n) = \frac{67}{91} \approx 0.736263$

\medskip
\noindent\textbullet~б) $n = 11$, $Q(n) = \frac{451}{455} \approx 0.991208$

\medskip
\noindent\textbullet~в) $n = 21$, $Q(n) = 1$
\end{subtask}

\begin{subtask}{2}
Подставим для a), б), в) при $M = 10$:

\medskip
\noindent\textbullet~a) $n = 5$, $Q(n) = \frac{182594}{264845} \approx 0.689437$

\medskip
\noindent\textbullet~б) $n = 11$, $Q(n) = \frac{438024217}{466921735} \approx 0.938111$

\medskip
\noindent\textbullet~в) $n = 21$, $Q(n) = \frac{578896}{580027} \approx 0.998050$

\end{subtask}

%---------------------------------------------------------------------------------------------------
\medskip
\begin{task}{9}
Найти вероятность того, что в $n$ испытаниях схемы Бернулли с вероятностью успеха $p$ появятся $m + l$ 
успехов, причем $l$ успехов появлятся в $l$ последних испытаниях.
\end{task}

\begin{solution}
Найдем вероятность события как произведения двух независимых событий: в первых $n - l$ испытаниях появится $m$ успехов,
в последних  $l$ испытаниях появится $l$ успехов.
\[
  P(AB) = P(A) \cdot P(B) = \left(C_{n - l}^m \cdot p^m \cdot (1 - p)^{n - l - m}\right) \cdot 
  \left(1 \cdot p^l\right) = C_{n - l}^m \cdot  p^{m + l} \cdot (1 - p)^{n - l -m }
.\] 
\end{solution}

\begin{result}
$C_{n - l}^m \cdot  p^{m + l} \cdot (1 - p)^{n - l -m }$.
\end{result}

%---------------------------------------------------------------------------------------------------
\medskip
\begin{task}{10}
Двое бросают монету по $n$ раз. Найти вероятность того, что у них выпадет одинаковое число гербов.
\end{task}

\begin{solution}
Решим несколькими способами, в первом получим ответ в виде суммы $n + 1$ слагаемых, во втором простое выражение.

\medskip
\noindentПусть монету просают $2n$ раз, причем в первом случае выпало $k$ гербов и во втором случае столько же. Тогда
\[
  \sum_{k = 0}^n C_n^k \cdot \left(\frac{1}{2}\right)^k \cdot \left(\frac{1}{2}\right)^{n - k} \cdot 
  C_n^k \cdot \left(\frac{1}{2}\right)^k \cdot \left(\frac{1}{2}\right)^{n - k} = 
  \sum_{k = 0}^n \left(C_n^k\right)^2 \cdot  \left(\frac{1}{2}\right)^{2n} =  \left(\frac{1}{2}\right)^{2n} \sum_{k = 0}^n \left(C_n^k\right)^2
.\] 

\bigskip
\noindentВторой способ: пусть один игрок выбросил $k$ гербов из $n$ бросков и  второй выбросил столько же. Давайте
инвертируем результаты испытаний второго игрока и будем считать герб за решку и наоборот. Тогда всего должно быть выброшено $n$ гербов из $2n$ испытаний.
Найдем чему это равно
\[
  C_{2n}^n \left(\frac{1}{2}\right)^n \left(\frac{1}{2}\right)^n = \left(\frac{1}{2}\right)^{2n} C_{2n}^n
.\] 
\end{solution}

\begin{result}
$\left(\frac{1}{2}\right)^{2n} C_{2n}^n$.
\end{result}

%---------------------------------------------------------------------------------------------------
\medskip
\begin{task}{11}
Из множества $S = \left\{1, 2, \ldots, n\right\}$ выбирается подмножество $A_1$ так, что каждый 
элемент из $S$ независимо от остальных с вероятностью $p$ включается в множество $A_1$ и с вероятностью
$q = 1- p$ не включается. Аналогичным образом независимо от $A_1$ выбирается подмножество $A_2$. Найти вероятности событий: а)
$A_1 \bigcap A_2 = \emptyset$; б) множество $A_1 \bigcap A_2$ состоит из $k = \left(k = 0, 1, 2, \ldots, n\right)$ 
элементов; в) $\abs{A_1} > \abs{A_2}$, $q = p = \frac{1}{2}$.
\end{task}

\begin{subtask}{а)}
 Выбираем множество $A_1$ :  $\sum_{k = 0}^n C_n^k p^k q^{n -k }$. Теперь выберем второе множество так,
 чтобы те элементы, которые были выбраны для первого множества, они не были выбраны для второго множества,
 остальные элементы можно выбирать любым образом.  Тогда общая вероятность:  
 \[
   \sum_{k = 0}^n C_n^k p^k q^{n - k} q^k = \sum_{k = 0}^n C_n^k p^k q^n = q^n \sum_{k = 0}^n C_n^k p^k = q^n (1 + p)^n = (1 - p)^n \cdot (1 + p)^n = (1 - p^2)^n
 .\] 
\end{subtask}


\begin{subtask}{б)}
Воспользуемся пунктом а). Для того, чтобы пересечение первого и второго множество было равно k (его мощность), необходимо чтобы в $n - k$ местах множества
не  пересекались, а в $k$ местах пересекались. Воспользовавшись результатом из пункта а) получим:
\[
  \left(1 - p^2\right)^{n - k} \cdot p^{2k} \cdot C_n^k
.\] Другими словами, мы выбрали подмножество множества $S$, состоящее из $n - k$ элементов, которое является пустым, и оставшееся множество, которое
полностью состоит из взятых элементов первым и вторым множеством. Данное множество можно выбрать $C_n^k$ способами.
\end{subtask}

\begin{subtask}{в)}
Также воспользуемся пунктом а). Из задачи 10 мы выяснили, что вероятность того, что 2 множества равны по мощности (выпало одинакое количество гербов)
есть $\left(\frac{1}{2}\right)^{2n} C_{2n}^n$. Тогда вероятность того, что множества не равны - $1 - \left(\frac{1}{2}\right)^{2n} C_{2n}^n$. Теперь
поскольку требуется, чтобы первое множество было по мощности больше чем второе, то поделим данный результат на 2, поскольку по способу набора элементов
данные два множества эквивалентны. Результат: 
\[
  P(\abs{A_1} > \abs{A_2}) = \frac{1 - \left(\frac{1}{2}\right)^{2n} C_{2n}^n}{2}
.\] 
\end{subtask}

\begin{result}
а) $(1 - p^2)^n$ б) $\left(1 - p^2\right)^{n - k} \cdot p^{2k} \cdot C_n^k$ в) $\frac{1 - \left(\frac{1}{2}\right)^{2n} C_{2n}^n}{2}$.
\end{result}


\begin{task}{19}
Вероятность попадания в цель при каждом выстреле равна 0.001. Найти вероятность попадания в цель 
двумя и более выстрелами при залпе в 5000 выстрелов.
\end{task}

\begin{solution}
Воспользуемся методом Пуассона, так как $\lambda = np = 5$, $p << 1$. Вычислим вероятность попадания двух и
более выстрелов как $1 - P(m \in \{0,1\})$
\[
  P(\mu_n) \approx \frac{\lambda^m_n}{m!}e^{-\lambda_n}, \; P(\mu_n \ge 2) \approx 1 - (\frac{5^0}{0!})e^{-5} - (\frac{5^1}{1!})e^{-5} = 0.9595723\ldots
.\] 
\end{solution}

\begin{result}
$0.9595723\ldots$.
\end{result}

%---------------------------------------------------------------------------------------------------

