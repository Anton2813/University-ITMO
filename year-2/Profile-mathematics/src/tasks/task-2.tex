\subsection*{Основные определения}

\noindent \textasteriskcentered~\textit{Минор $k$-го порядка матрицы} - определитель матрицы, составленный на пересечении $k$ строк и $k$ столбцов.

\smallskip
\noindent \textasteriskcentered~\textit{Ранг матрицы} - наивысший отличный от нуля минор порядка.

\smallskip
\noindent \textasteriskcentered~Минор, порядок которого определяет ранг матрицы, называется \textbf{\textit{базисным}}.

\medskip
\noindent \textbullet~Очевидно, что у матрицы $A_{m \times n}$ ранг будет равен: $0 \le r \le min(m;n)$.

\subsection*{Свойства}

\noindent \textbullet~При транспонировании матрицы ее ранг не меняется.

\smallskip
\noindent \textbullet~Если вычеркнуть из матрицы нулевой ряд, то ранг матрицы не изменится.

\smallskip
\noindent \textbullet~Ранг матрицы не изменяется при элементарных преобразованиях.
