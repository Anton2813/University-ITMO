\documentclass{article}

%----geometry--------------------------------------------------------------------------------------
\usepackage{geometry}

% total - determines printable width, height
\geometry{ 
	a4paper, total={180mm,267mm}
}

%----text,fonts------------------------------------------------------------------------------------
\usepackage[T2A]{fontenc}
\usepackage[russian,english]{babel}
\usepackage[utf8]{inputenc}
\usepackage{setspace}

% \setstretch{0,9}

%----ToC-------------------------------------------------------------------------------------------
\usepackage{blindtext}
\usepackage{hyperref}

% table of contents
\addto\captionsenglish{
	\renewcommand{\contentsname}
	{Оглавление}
}

\hypersetup{
	allcolors=black
}

%----math,graphics---------------------------------------------------------------------------------
\usepackage{amsmath,amsfonts,amssymb}
\usepackage{amsthm}

\usepackage{graphicx}
\graphicspath{{images/}}

\usepackage{wrapfig}
\usepackage{tabularx}

% environments
\renewenvironment{proof}{\medskip\noindent\textbf{Доказательство}.}{\hfill$\blacksquare$}

% roman caps symbol
\newcommand{\romannumeralcaps}[1]
    {\MakeUppercase{\romannumeral #1}}
% norm symbol%
\newcommand{\norm}[1]{\left\lVert#1\right\rVert}
% module
\newcommand{\abs}[1]{\left\lvert#1\right\rvert}

%----title-----------------------------------------------------------------------------------------
\title{Профильная математика\\ Билеты к экзамену 3 семестр}
\date{2021-2022}
\author{}

%----document--------------------------------------------------------------------------------------
\begin{document}
\maketitle

\begin{sloppypar}
\tableofcontents

\newpage
\section{Матрицы и их основные свойства.}
\subsection*{Основные определения}

\noindent \textasteriskcentered~\textit{Матрица} - прямоугольная таблица чисел, содержащая $m$ строк одинаковой длины (или $n$ столбцов одинаковой длины).
\[
A = 
\begin{pmatrix}
    a_{11} & a_{12} & \dots & a_{1n} \\
    a_{21} & a_{22} & \dots & a_{1n} \\
    \vdots & \vdots & \ddots & \vdots \\
    a_{m1} & a_{m2} & \dots & a_{mn}
\end{pmatrix}
\]

\noindent \textasteriskcentered~Матрицу $A$ называют матрицей \textit{размера $m \times n$} и пишут $A_{m \times n}$. 
Числа $a_{ij}$ называются ее \textit{элементами}. 
Элементы, стоящие на диагонали, идущей из верхнего левого угла, образуют \textit{главную диагональ}. Из нижнего левого -- \textit{побочная диагональ}.

\smallskip
\noindent \textasteriskcentered~Матрицы \textbf{\textit{равны между собой}}, если равны все соответствующие элементы этих матриц ($a_{ij} = b_{ij}$).

\smallskip
\noindent \textasteriskcentered~\textit{Квадратная матрица}: ($m = n$)

\smallskip
\noindent \textasteriskcentered~\textit{Матрица строка}:
$\begin{pmatrix}
    a_{11} & a_{12} & \dots & a_{1n}
\end{pmatrix}$

\smallskip
\noindent \textasteriskcentered~\textit{Матрица столбец(вектор)}: 
$\begin{pmatrix}
    a_{11} \\ 
    a_{21} \\
    \vdots \\
    a_{m1}
\end{pmatrix}$

\smallskip
\noindent \textasteriskcentered~\textit{Диагональная матрица}: 
$\forall a_{ij} : a_{ij} = 0 \; \forall i \neq j$

\smallskip
\noindent \textasteriskcentered~\textit{Единичная матрица}:
$E_{n \times n} = 
\begin{pmatrix}
    1 & 0 & \dots & 0 \\
    0 & 1 & \dots & 0 \\
    \vdots & \vdots & \ddots & \vdots \\
    0 & 0 & \dots & 1
\end{pmatrix}
$

\smallskip
\noindent \textasteriskcentered~\textit{Нулевая матрица}:
$O_{m \times n} = 
\begin{pmatrix}
    0 & 0 & \dots & 0 \\
    0 & 0 & \dots & 0 \\
    \vdots & \vdots & \ddots & \vdots \\
    0 & 0 & \dots & 0
\end{pmatrix}
$

\smallskip
\noindent \textasteriskcentered~\textit{Треугольная матрица}:
$A_{n \times n} = 
\begin{pmatrix}
    a_{11} & a_{12} & \dots & a_{1n} \\
    0 & a_{22} & \dots & a_{2n} \\
    \vdots & \vdots & \ddots & \vdots \\
    0 & 0 & \dots & a_{mn}
\end{pmatrix}$

\smallskip
\noindent \textasteriskcentered~\textit{Транспонированная матрица}:
$A = 
\begin{pmatrix}
    a_{11} & a_{12} & \dots & a_{1n} \\
    a_{21} & a_{22} & \dots & a_{2n} \\
    \vdots & \vdots & \ddots & \vdots \\
    a_{m1} & a_{m2} & \dots & a_{mn}
\end{pmatrix},\;
A^T = 
\begin{pmatrix}
    a_{11} & a_{21} & \dots & a_{m1} \\
    a_{12} & a_{22} & \dots & a_{m2} \\
    \vdots & \vdots & \ddots & \vdots \\
    a_{1n} & a_{2n} & \dots & a_{mn}
\end{pmatrix}$

\medskip
\noindent \textasteriskcentered~При помощи элементарных преобразований любую матрицу можно привести к такой, у которой в начале главной диагонали будет стоять некоторое количество единиц, а затем будут идти нули, не на главной диагонали также нули. Такую матрицу называют \textbf{\textit{канонической}}.

\subsection*{Действия над матрицами}

\noindent \textasteriskcentered~\textit{Матрицы равны}: $A_{m \times n} = B_{m \times n}, \; if \; a_{ij} = b_{ij} 
\;\forall i = \overline{1,\dots m}, j = \overline{1, \dots n}$

\smallskip
\noindent \textasteriskcentered~\textit{Суммой} $A_{m \times n} = (a_{ij})$ и $B_{m \times n} = (b_{ij})$, называется $C_{m \times n} = (c_{ij})$ $: \forall c_{ij} = (a_{ij} + b_{ij})$.

\smallskip 
\noindent \textasteriskcentered~\textit{Произведение на число}: $A_{m \times n} = (a_{ij})$ на $\lambda \in \mathbb{R}$ называется матрица $B_{m \times n} = (\lambda \cdot a_{ij})$.


\subsection*{Основные свойства}

\noindent 1) Коммутативность: 
$A_{m \times n} + B_{m \times n} = B_{m \times n} + A_{m \times n}$

\smallskip 
\noindent 2) Ассоциативность:
$(A_{m \times n} + B_{m \times n}) + C_{m \times n} = A_{m \times n} + (B_{m \times n} 
+ C_{m \times n})$

\smallskip 
\noindent 3) Существование нулевой матрицы:
$(A_{m \times n} + O_{m \times n} = A_{m \times n})$

\smallskip 
\noindent 4) Умножение на число:
$\lambda (A + B) = \lambda A + \lambda B$

\smallskip 
\noindent 5) Дистрибутивность: 
$(\alpha + \beta) A = \alpha A + \beta A$

\smallskip 
\noindent 6) Единичный элемент:
$E * A = A$

\smallskip 
\noindent 7) $A - O = A$

\smallskip 
\noindent 8) $\alpha (\beta A) = (\alpha \beta) A$


\subsection*{Элементарные преобразования}

\noindent \textbullet~Перестановка местами двух параллельных рядов матрицы.

\noindent \textbullet~Умножение всех элементов ряда матрицы на число, отличное от нуля.

\noindent \textbullet~Прибавление ко всем элементам ряда матрицы соответствующих элементов параллельного ряда, умноженных на одно и то же число.

\smallskip
\noindent \textasteriskcentered~Две матрицы называются \textbf{\textit{эквивалентными}}, если одна из другой
получается с помощью элементарных преобразований: $A \sim B$


\subsection*{Прочее}

\noindent \textasteriskcentered~Матрицы $A$ и $B$ называются \textbf{\textit{перестановочными}}, если $A B = B A$.


\subsubsection*{Свойства, связанные с умножением матриц}

\noindent 1) $A \cdot (B \cdot C) = (A \cdot B) \cdot C$

\smallskip
\noindent 2) $A \cdot (B + C) = A B + A C$

\smallskip
\noindent 3) $(A + B) \cdot C = A C + B C$

\smallskip
\noindent 4) $\lambda (A B) = (\lambda A) B = A \cdot (\lambda B)$

\smallskip
\noindent 5) $A \cdot B \neq B \cdot A$

\smallskip
\noindent 6) $det (A \cdot B) = det(A) \cdot det(B)$


\subsubsection*{Свойства транспонирования}

\noindent 1) $(A + B)^T = A^T + B^T$

\smallskip
\noindent 2) $(AB)^T = B^T \cdot A^T$

\section{Ранг матрицы. Свойства.}
\subsection*{Неравенства Гельдера и Минковского.}
\noindent \textbf{Неравенство Гельдера.} \textit{Если $f(t) \in L_p(a, b)$, $p > 1$ и $g(t) \in L_q(a, b)$, где $\dfrac{1}{p} + \dfrac{1}{q} = 1$, то произведение 
$f(t)$ и $g(t)$ - суммируемая на $[a, b]$ функция и } 
\[
    \int_a^b \abs{f(t) g(t)} dt \le \left(\int_a^b \abs{f(t)}^p dt\right)^{\frac{1}{p}} \left(\int_a^b \abs{g(t)}^q dt\right)^{\frac{1}{q}}
\]

\medskip 
\noindent \textbf{Неравенство Минковского.} \textit{Если $f(t), g(t) \in L_p(a, b)$, то} 
\[
    \left(\int_a^b \abs{f(t) + g(t)}^p dt\right)^{\frac{1}{p}} \le 
    \left(\int_a^b \abs{f(t)}^p dt \right)^{\frac{1}{p}} + \left( \int_a^b \abs{g(t)}^p dt\right)^{\frac{1}{p}}
\]

\subsection*{Пространство последовательностей $l_p$.}

\noindent \textbullet~$l_p, \; p \ge 1$;
$l_p = \{\overline{x} = (x_1, x_2, \dots) : \sum_{n = 1}^{\infty} \abs{x_n}^p < +\infty\}$

\begin{proofexpr*}
    $\overline{x}$, $\overline{y} \in l_p \Rightarrow \overline{x} + \overline{y} \in l_p$
\end{proofexpr*}
\begin{proof}
\par \noindent \textbullet~$\overline{x} + \overline{y} = (x_1 + y_1, x_2 + y_2, \dots)$.

\smallskip
\noindent \textbullet~$\sum_{n = 1}^{\infty} \abs{x_n + y_n}^p \le 
\sum (\abs{x_n} + \abs{y_n})^p = 
\sum_{n : \abs{x_n} \ge \abs{y_n}} + \sum_{n : \abs{x_n} < \abs{y_n}} \le 
2^p \sum_{n : \abs{x_n} \ge \abs{y_n}} \abs{x_n}^p + 2^p \sum_{n : \abs{x_n} < \abs{y_n}} \abs{y_n}^p \le
2^p \left(\sum_{n = 1}^{\infty} \abs{x_n}^p + \abs{y_n}^p \right) < +\infty$ 

\medskip
\noindent \textbullet~Последнее выражение конечно по условию: $\sum_{n = 1}^\infty \abs{x_n}^p < +\infty$

\end{proof}


\begin{proofexpr*}
$\overline{x} \in l_p \Rightarrow \alpha \overline{x} \in l_p$
\end{proofexpr*}

\begin{proof}
\par\noindent \textbullet~$\alpha \overline{x} = (\alpha x_1, \alpha x_2, \dots)$. 

\smallskip
\noindent \textbullet~$\sum_{n = 1}^{\infty} \abs{\alpha x_n}^p =
\abs{\alpha}^p \sum_{n = 1}^{\infty}\abs{x_n}^p < +\infty$
\end{proof}

\subsection*{Пространство непрерывно-дифференцируемых функций $C[a, b]$.}

\noindent \textbullet~$C^{(p)}[a, b], \; p = 0, 1, 2, 3 \dots$;
$C^{(p)}[a, b] = \{ f : [a, b] \rightarrow \mathbb{R}$, которые $p$ раз непрерывно дифф на отрезке $[a, b]$$\}$

\medskip
\noindent \textbullet~При $p = 0$ $f^{(0)} = f$, по определению.

\smallskip
\noindent \textbullet~$(f + g)^{(k)} = f^{(k)} + g^{(k)}$.

\section{Обратная матрица. Теорема существования обратной матрицы.}
\subsection*{Невырожденная матрица.}

\noindent \textasteriskcentered~Квадратная матрица называется \textit{невырожденной}, если определитель $\Delta = det A$ не равен нулю: $\Delta = det A \neq 0$. В противном случае матрица называется \textit{вырожденной}.

\subsection*{Союзная матрица.}

\noindent \textasteriskcentered~Матрицей, \textbf{\textit{союзной}} к матрице $A$, называется матрица 
\[
    A^* = 
    \begin{pmatrix}
        A_{11} & A_{21} & \dots & A_{n1} \\
        A_{12} & A_{22} & \dots & A_{n2} \\
        \vdots & \vdots &\ddots & \vdots \\
        A_{1n} & A_{2n} & \dots & A_{nn}
    \end{pmatrix}
\]
, где $A_{ij}$ - алгебраическое дополнение элемента $a_{ij}$ данной матрицы.

\subsection*{Обратная матрица.}

\noindent \textasteriskcentered~Матрица $A^{-1}$ называется \textbf{\textit{обратной}} матрице $A$, если выполнено условие $A \cdot A^{-1} = A^{-1} \cdot A = E$

\bigskip
\noindent \textbf{Теорема}. Всякая невырожденная матрица имеет обратную. ($\exists A^{-1} \Longleftrightarrow detA \neq 0$)

\begin{proof}
\par \noindent \textit{Необходимость}. Пусть $\exists A^{-1} \Rightarrow A \cdot A^{-1} = E$. Так как $det(A \cdot A^{-1}) = det A * det A^{-1} \neq 0 \; (det E = 1 => det A \neq 0)$.

\smallskip
\noindent\textit{Достаточность}. Пусть $det A \neq 0$. Рассмотрим $A \cdot (A^{*})^T$.

\[
   A \cdot A^{*} = 
   \begin{pmatrix}
        A_{11} \cdot a_{11} + A_{12} \cdot a_{12} + \dots + A_{1n} \cdot a_{1n} & 
        0 & \dots & 0 \\ 
        0 & A_{21} \cdot a_{21} + \dots + A_{2n} \cdot a_{2n} &
        \dots & 0 \\ 
        \vdots & \vdots & \ddots & \vdots \\
        0 & 0 & \dots & 
        A_{n1} \cdot a_{n1} + \dots + A_{nn} \cdot a_{nn}  
   \end{pmatrix} = 
\]

$= det A \cdot E \Rightarrow \dfrac{A \cdot (A^*)^T}{det A} = E$.

\end{proof}
\section{Решение системы линейных уравнений методом Гаусса.}
\subsection*{Основные определения.}

\noindent \textasteriskcentered~Системой линейных алгебраических уравнений, содержащей m уравнений и n неизвестных, называется система вида

\begin{equation*}
    \begin{cases}
        a_{11} x_1 + a_{12} x_2 + \dots + a_{1n} x_n = b_1 \\
        a_{12} x_1 + a_{22} x_2 + \dots + a_{1n} x_n = b_2 \\
        \cdots \cdots \cdots \cdots \cdots \cdots \cdots \cdots \cdots \cdots \cdots \\
        a_{1n} x_1 + a_{2n} x_2 + \dots + a_{nn} x_n = b_n   
    \end{cases},
\end{equation*}

\noindent где числа $a_{ij}$ - \textit{коэффициенты} системы, числа $b_{ij}$ - \textit{свободные члены}.

\smallskip
\noindent Матричная форма записи: $A \cdot X = B$, где $A$ - \textit{основная матрица}, $X$ - вектор-столбец неизвестных $x_j$, $B$ - вектор-столбец свободных членов.

\smallskip
\noindent \textasteriskcentered~\textit{Расширенная} матрица - матрица $\overline{A}$, дополненная справа столбцом свободных членов.

\noindent \textbullet~Всякое решение можно записать в виде матрицы-столбца:
$
    C = 
    \begin{pmatrix}
        c_1 \\
        c_2 \\
        \vdots \\
        c_n
    \end{pmatrix}
$

\medskip\noindent \textasteriskcentered~Система уравнений называется \textbf{\textit{совместной}}, если она имеет хотя бы одно решение, и \textbf{\textit{несовместной}}, если она не имеет ни одного решения. 

\medskip\noindent \textasteriskcentered~Совместная система называется \textbf{\textit{определенной}}, если она имеет единственное решение, и \textbf{\textit{неопределенной}}, если она имеет более одного решения. 

\medskip\noindent \textasteriskcentered~Система линейных уравнений называется \textbf{\textit{однородной}}, если все ее свободные члены равны 0. 

\medskip\noindent \textasteriskcentered~\textit{Тривиальное }или \textit{нулевое} решение - $x_1 = x_2 = \dots = x_n = 0$ 


\subsection*{Решение системы линейных уравнений методом Гаусса.}

\noindent \textbullet~Для решения системы линейных уравнений по методу Гаусса необходимо систему привести к \textit{ступенчатому}, в частности \textit{треугольному} виду.

\begin{equation*}
    \begin{cases}
        a_{11} x_1 + a_{12} x_2 + \cdots + a_{1k} x_k + \cdots + a_{1n} x_n = b_1 \\
        \phantom{a_11 x_1 +~} a_{b2} x_2 + \cdots + a_{2k} x_k + \cdots + a_{2n} x_n = b_2 \\
         \phantom{a_11 x_1 +~} \cdots \cdots \cdots \cdots \cdots \cdots \cdots \cdots \cdots \cdots \cdots \cdots \\
         \phantom{a_{11} x_1 + a_{12} x_2 + \cdots + } a_{kk} x_k + \cdots + a_{kn} x_n = b_n \end{cases}
\end{equation*}

\noindent \textbullet~Затем свободные члены ($a_{i, k+1} x_{k_1}, \dots, a_{i, n} x_{n}$), если система не треугольная, перемещаем в правую часть и решаем обратным ходом, просто принимая в качестве иксов какие-то параметры ($\alpha, \beta, \gamma \dots$). Если система является треугольной, то свободных членов не будет, а только главные.
\section{Решение системы линейных уравнений методом Крамера.}
test
\section{Решение системы линейных уравнений матричным способом.}
\subsection*{Размерность.}

\noindent \textasteriskcentered~Пусть $X$ - линейное многообразие, $e_1, \dots, e_n$ - линейно-независимые векторы в $X : V(e_1, \dots, e_n) = \{ \sum_1^n 
\alpha_k e_k\}$\footnote{Чисто алгебраическое определение.}, где $V$ - линейная оболочка. Размерность $dim X = n$. Размерность отрезка = $1$, размерность квадрата - $2$.

\medskip
\noindent \checkmark~Есть еще понятие топологической размерности, но эта тема крайне трудная. Работы принадлежат Урысону (1930-е годы).

\bigskip
\begin{theorem*}[Фердинанд Рисс]
Пусть $dim X < +\infty \Rightarrow$ все нормы в $X$ эквивалентны.
\end{theorem*}

\begin{proof}

\noindent \textbullet~По условию в $X \; \exists e_1, \dots, e_n$ - линейно-нез. $: X = V(e_1, \dots, e_n)$, $\forall x \in X \Rightarrow$ Единственно($!$) $x = \sum_1^n 
\alpha_k e_k$.

\smallskip
\noindent \textbullet~Иксу соответствуют числа $x \leftrightarrow (\alpha_1, \dots, \alpha_n)$. Пусть $\norm{x}$ - норма в $X$. Помимо нее определим $\norm{x}_0$ -
как евклидовскую ($\sqrt{\sum_{k = 1} ^n \alpha_k^2}$).
Если исходная норма будет эквивалентна той, которую мы построили, то по транзитивности этого отношения эквивалентности любые две нормы на $x$ 
окажутся эквивалентными.

\smallskip 
\noindent \textbullet~По неравенству треугольника $\norm{x} \le \sum_1^n \abs{\alpha_k} \norm{e_k}$. По неравенству Гельдера, где $p = 2$: 
$\sum_1^n \abs{\alpha_k} \norm{e_k} \le \sqrt{\sum_{k = 1}^{n} \norm{e_k}^2} \cdot \sqrt{\sum_{k = 1}^n \abs{\alpha_k}^2}$. 

\smallskip
\noindent \textbullet~$\sqrt{\sum_{k = 1}^n \abs{\alpha_k}^2} = \norm{x}_0$, $ \sqrt{\sum_{k = 1}^{n} \norm{e_k}^2}$ - некоторая константа, обозначим $b$. Таким образом,
$\norm{x} \le b \norm{x}_0$.

\medskip 
\noindent \textbullet~Проверим $a \norm{x}_0 \le \norm{x}$. Для этого рассмотрим в $\mathbb{R}^n$ $f(\alpha_1, \dots, \alpha_n) = \norm{\sum_{k = 1}^n \alpha_k e_k}$.
Проверим, что $f$ - непрерывна в $\mathbb{R}^n$.

\smallskip
\noindent \textbullet~$\abs{f(\overline{\alpha} + \Delta \overline{\alpha}) - f(\overline{\alpha})} = \abs{\norm{\sum \alpha_k e_k + \sum \Delta\alpha_k e_k} - 
\norm{\sum \alpha_k e_k}} \le \norm{\sum \Delta \alpha_k e_k} \le \sum \abs{\Delta \alpha_k} \norm{e_k} \le b \cdot \norm{\Delta \overline{\alpha}}_0$.
Получили $\abs{\Delta f(\overline{\alpha})} \le b \cdot \norm{\Delta \overline{\alpha}}_0 \Rightarrow f$ - непрерывна в $\mathbb{R}^n$.

\smallskip 
\noindent \textbullet~Рассмотрим единичную сферу в $\mathbb{R}^n$ : $S_1 = \{ \overline{\alpha} : \sum_{k = 1}^n \alpha_k^2 = 1\}$. В силу непрерывности $f$ по 
т. Вейерштрасса в матанализе об экстремальных значениях непрерывной функции $\exists \overline{\alpha}^* \in S_1 : f(\overline{\alpha}^*) = 
\min_{\overline{\alpha } \in S} f(\overline{\alpha}) = a$.

\smallskip 
\noindent \textbullet~Если допустить, что $a = 0$, то $f(\overline{\alpha}^*) = 0$, а тогда по формуле для $f$ $\norm{\sum \alpha^*_k e_k} = 0 \Rightarrow
\sum \alpha^*_{k} e_k = 0$, а по линейной независимости $e_k \Rightarrow$ все $\alpha_k^* = 0$, а тогда эта точка не будет принадлежать сфере $\overline{\alpha}^* \notin
S_1$, что противоречит тому, что мы брали точку на сфере. То есть $a > 0$.

\smallskip
\noindent \textbullet~$\forall x \in X$, $ x = \sum \alpha_k e_k$. Рассмотрим соответствующее значение функции $f$ на этих коэффициентах: $f(\overline{\alpha}) = 
\norm{\sum \alpha_k e_k}$. Пусть $\beta_k = \dfrac{\alpha_k}{\norm{\overline{\alpha}}_e}$, $\sum \beta^2_k = \sum \dfrac{\alpha^2_k}{\norm{\overline{\alpha}}^2_e} = 1$
 т.е $\overline{\beta} = (\beta_1, \dots, \beta_n) \in S_1$. Тогда $f(\overline{\beta}) \ge a$.

\smallskip 
\noindent \textbullet~Если записать тождество $f(\overline{\alpha}) = \norm{\overline{\alpha}}_e \cdot \norm{\sum \dfrac{\alpha_k}{\norm{\overline{\alpha}}_e} e_k} 
= \norm{\overline{\alpha}}_e \cdot \norm{\sum \beta_k e_k}
\ge \norm{\overline{\alpha}}_e \cdot a = \norm{x}_0 \cdot a$. $f(\overline{\alpha}) = \norm{x}$. Таким образом, получаем $\norm{x} \ge a \cdot \norm{x}_0$.
\end{proof}
\section{Теорема Кронекера-Капелли.}
test
\section{Критерий линейной зависимости строк (столбцов) матрицы.}
test
\section{Теорема о базисном миноре.}
test
\section{Свойства системы линейных однородных алгебраических уравнений.}
\noindent \textasteriskcentered~Пусть $\overline{V}_n = \overline{V}_{r_n}(a_n) = \{ \norm{x - a_n} \le r_n\}$. Если последующий шар содержится в предыдущем
$\overline{V}_{n + 1} \subset \overline{V}_{n}$, то тогда данная система называется \textit{вложенной}.  

\medskip
\noindent \textbf{Утверждение. }\textit{Пусть $\overline{V}_1 = \overline{V}_{r_1}(a_1)$, $\overline{V}_2 = \overline{V}_{r_2}(a_2)$ - замкнутые шары в $X$. Тогда $
\overline{V}_1 \subset \overline{V}_2 \Longleftrightarrow \norm{a_2 - a_1} \le r_2 - r_1$}. 

\begin{theorem*}[Принцип вложенных шаров]
Пусть $X$ - В-пространство и пусть система шаров $\{ \overline{V}_n = \overline{V}_{r_n}(a_n)\}$ - вложенная. Тогда их пересечение не пустое $\bigcap\limits_1^\infty 
\overline{V}_n \neq \emptyset$. 
\end{theorem*}

\begin{proof}
\par\noindent \textbullet~По вложенности и предыдущему утверждению $\norm{a_n - a_{n + 1}} \le r_n - r_{n + 1} \Rightarrow 0 \le r_{n + 1} \le r_n$, то есть 
последовательность $\{ r_n\}$ убывает и ограничена снизу, а тогда по т. Вейерштрасса о пределе монотонной последовательности $\exists r = \lim r_n$.

\noindent \textbullet~В силу вложенности можем написать неравенства с произвольными индексами: $\norm{a_m - a_{m + p}} \le r_m - r_{m+p} \to 0 \; m, p \to \infty$, 
а тогда последовательность центров $\{ a_n\}$ сходится в себе, а тогда по полноте $X$ $\exists a = \lim a_n$.

\noindent \textbullet~Опять таки в силу вложенности $\overline{V}_{m + p} \subset \overline{V}_m \Rightarrow a_{m + p} \in \overline{V}_m$, который является замкнутым 
множеством, поэтому если $p \to \infty$, то $a_{m + p} \to a \Rightarrow a \in \overline{V}_m$, а тогда в силу произвольности $m$: $a \in \bigcap\limits_1^\infty 
\overline{V}_m$.
\end{proof}
\section{Теорема о фундаментальной системе решений системы линейных однородных алгебраических уравнений.}
test
\section{Свойства решений системы линейных неоднородных алгебраических уравнений.}
test
\section{Определение линейного пространства. Свойства линейного пространства. }
test
\section{Базис линейного пространства. Теорема о разложении элемента линейного пространства по базису.}
test
\section{Размерность линейного пространства. Теоремы о связи базиса и размерности.}
\subsection*{Определение гильтертового пространства.}

\noindent \textasteriskcentered~Полное, бесконечномерное, унитарное пространство назывется \textit{пространством Гильберта} или \textit{гильбертовым пространством}.

\smallskip
\noindent \textasteriskcentered~$H$ - гильбертовое пространство, пространство в котором определено скалярное произведение $(x, y)$, которое позволяет говорить об 
ортогональности точек, 
понятие нормы $\norm{x} = \sqrt{(x, x)}$ и пространство является полным или Банаховым ($x_n - x_m \to 0 \Rightarrow \exists \lim x_n$).

\bigskip
\noindent \textasteriskcentered~Пусть $H_1$ - подпростраство в $H$. Пусть $A \subset H$, тогда $A^{\bot} = \{ x \in H : x \bot a, \forall a \in A\}$.
Получившееся множество, проходящее через нуль, называется \textit{ортогональным дополнением} для множества $A$. $A$ - может быть любым, однако $A^\bot$ всегда подпространство $H$.

\begin{tikzpicture}[scale=0.2]
    \draw[dashed] (-10, 0) -- (10, 0);
    \draw[thick] (2, 0) --node[anchor=north]{$A$} (7, 0);
    \draw[solid] (0, 5) --node[anchor=east, near end]{$A^\bot$} (0, -5);
\end{tikzpicture}


\subsection*{Основная теорема теории гильбертовых пространств.}

\begin{theorem*}[Основная теорема теории гильбертовых пространств]
   Пусть $H_1$ - подпространство $H$, тогда $\forall x \in H$ $\exists$ единственные $x_1 \in H_1$ и $x_1^\bot \in H_1^\bot : x = x_1 + x_1^\bot$, то есть можно 
   разложить с сумму взаимноортогональных точек.
\end{theorem*}

\begin{proof}
\smallskip
\par\noindent \textbullet~Докажем единственность. Пусть $x'_1 \in H_1, \; x_1^{\bot'} \in H_1^\bot : x = x_1' + x_1^{\bot'}, \; x = x_1 + x_1^\bot \Rightarrow
x_1 - x_1' = x_1^{\bot'} - x_1^\bot$. Так как $H$ - подпространство, то $x_1 - x_1' \in H_1$, $x_1^{\bot'} - x_1^{\bot} \in H_1^\bot$. Так как $H_1^\bot$ - 
ортогональное дополнение $\Rightarrow H_1^\bot \cap H_1 = {0}$ - тривиальное, а тогда из последнего равенства $x_1 - x_1' = x_1^{\bot'} - x_1^\bot = 0$ получаем, что 
эти точки равны. 

\medskip 
\noindent \textbullet~Так как $H_1$ - подпространство, то $H_1$ - выпуклое замкнутое множество, тогда по теореме из пункта 17 $\forall x \in H \; \exists$ минимизирующий
элемент $x_1 \in H_1$, значит $\forall y \in H_1 \Rightarrow \norm{x - x_1} \le \norm{x - y}$. Обозначим $x_1^\bot = x - x_1$ и проверим, что $x_1^\bot \in H_1^\bot$. 

\smallskip
\noindent \textbullet~$\forall y \in H_1$, $\forall t \in \mathbb{C}$. Тогда $x_1 + t y \in H_1$, потому что это подпространство, и в силу неравенства выше получим 
$\norm{x - x_1} \le \norm{x - (x_1 + t y)}$ - верно для всех $t$. Перейдя к обозначениям выше $\norm{x_1^\bot} \le \norm{x_1^\bot - t y}^2 = \norm{x_1^\bot}^2 - 
(x_1^\bot, ty) - (ty, x_1^\bot) + \norm{t y}^2$.

\smallskip
\noindent \textbullet~Сокращаем и получаем $\overline{t} (x_1^\bot, y) + t (y, x_1^\bot) \le \abs{t}^2 \norm{y}^2$. В частности, неравенство выполняется при 
$t = \dfrac{\overline{(y, x_1^\bot)}}{\norm{y}^2}$.

\[
    \dfrac{(y, x_1^\bot)(x_1^\bot, y)}{\norm{y}^2} + \dfrac{\overline{(y, x_1^\bot)}(y, x_1^\bot)}{\norm{y}^2} \le \dfrac{\abs{(y, x_1^\bot)}^2}{\norm{y}^4} \cdot 
    \norm{y}^2
\]

\[
    2 \abs{(y, x_1^\bot)} \le \abs{(y, x_1^\bot)}^2
\]
\noindent \textbullet~Если допустить, что $(y, x_1^\bot) \neq 0 \Rightarrow 2 \le 1$. Таким образом, $(y, x_1^\bot) = 0 \Rightarrow \forall y \in H_1 \Rightarrow 
y \bot x_1^\bot \Rightarrow x_1^\bot \in H_1^\bot$. Разложение получено.
\end{proof}

\bigskip
\noindent \textasteriskcentered~$x = x_1 + x_1^\bot$, $x_1$ - проекция $x$ на $H_1$. Верно для любых абстрактных гильбертовых пространств.

\begin{tikzpicture}[scale=0.2]
    \draw[solid] (-10, 0) --node[anchor=north, very near start]{$H_1$} (10, 0);
    \draw[solid] (0, -7) --node[anchor=east, very near start]{$H_1^\bot$} (0, 7);
    \draw[->] (0, 0) --node[anchor=north, very near end]{$x$} (8, 6);
    \draw[dotted](8, 6) -- (8, 0);
    \draw[dotted](8, 6) -- (0, 6);
    \draw[very thick, ->] (0, 0) -- node[anchor=north, very near end]{$x_1$} (8, 0);
    \draw[very thick, ->] (0, 0) -- node[anchor=east, very near end]{$x_1^\bot$} (0, 6);
\end{tikzpicture}


\subsection*{Ортогональные ряды.}

\noindent \textbullet~Пусть есть $\{ e_j \}$ - ОНС. Мы хотим понять, при каком условии $\forall x \; \norm{x}^2 = \sum \abs{(x, e_j)}^2$ - выполняется уравнение замкнутости.
Для этого рассмотрим ортогональный ряд $\sigma(x) = \sum_1^\infty (x, e_j) e_j$ и установим критерий сходимости. 

\bigskip 
\noindent\textbf{Утверждение \textnormal{(Критерий сходимости ортогонального ряда)}.} \textit{Пусть $H$ - гильбертовое пространство, $\sum_1^\infty x_j$ -
ортогональный ряд. Тогда он сходится $\Longleftrightarrow \sum_1^\infty \norm{x_j}^2 < +\infty$.} 

\begin{proof}

\smallskip
\noindent1)~Пусть ряд сходится $\Rightarrow \exists S = \lim_{n \to \infty} S_n$, $S_n = \sum_1^n x_j$. Из того, что $S_n \to S \Rightarrow \norm{S_n} \to \norm{S}$.
Сосчитаем квадрат нормы частичных сумм $\norm{S_n}^2 = \left(\sum_1^n x_j , \sum_1^n x_j\right) = \sum_{i, j = 1}^n (x_i, x_j)$, где если $i \neq j \Rightarrow = 0$.
Тогда $\norm{S_n}^2 = \sum_1^n \norm{x_j}^2 \to \norm{S}^2$. Поскольку $\sum_1^n \norm{x_j}^2$ - частичная сумма $\sum_1^\infty \norm{x_k}^2$, значит сходится. 

\medskip
\noindent \textbullet~Также в силу того, что есть соотношение $\sum_1^n \norm{x_j}^2 \to \norm{S}^2$ приходим, что $\norm{\sum_1^\infty x_j}^2 = \sum_1^\infty \norm{x_j}^2$
, что можно назвать теоремой Пифагора, для пространства гильберта.

\noindent2)~Пусть $\sum_1^\infty \norm{x_k}^2 < +\infty$. Тогда $\norm{S_{n+p} - S_n}^2 = \norm{\sum_{k = n + 1}^{n + p} x_k}^2$. Точно также как считалось выше за счет 
ортогональности $\norm{\sum_{k = n + 1}^{n + p} x_k}^2 = \sum_{k = n + 1}^{n+p} \norm{x_k}^2 \to 0$, $n, p \to \infty$ поскольку числовой ряд из квадратов норм сходится.
В результе получаем, что $\norm{S_{n +p} - S_n} \to 0$, а по полноте $H$ $\exists S = \lim S_n$. Доказано.
\end{proof}
\section{Подпространство линейного пространства. Свойства.}
test
\section{Линейный оператор. Матрица линейного оператора.}
\subsection*{Определение выпуклого и невыпуклого множества.}

\noindent \textasteriskcentered~Множество является \textit{выпуклым}, если $x, y \in M$, $\lambda \in [0, 1] \Rightarrow \lambda x + (1 - \lambda) y \in M$. В противном 
случае, множество называется \textit{невыпуклым}. Любое подпространство обязательно является выпуклым, однако обратное неверно.

\begin{tikzpicture}
    \draw (0, 0) .. controls (0, 1) and (1, 1) .. (1, 0.5)
                 .. controls (1, 0.25) and (2, 0.25) .. (2, 0.5)
                 .. controls (2, 1) and (3, 1) .. (3, 0.5)
                 .. controls (3, -1) and (0, -1) .. (0, 0);
    \draw (1.5, 0) node{невыпуклое};

    \draw (5, 0) .. controls (5, 1) and (8, 1) .. (8, 0)
                 .. controls (8, -1) and (5, -1) .. (5, 0);

    \draw (6.5, 0) node{выпуклое};


    \draw (12, 0) node{$M$} circle (1);
    \fill (14, 0) node[anchor=north]{$x$} circle[radius = 0.05];
    \draw (14, 0) -- (12, 0);
    \fill (13, 0) node[anchor=south west]{$\hat{x}$} circle[radius = 0.05];
\end{tikzpicture}


\subsection*{Теорема о наилучшем приближении в Н для случая выпуклого, замкнутого множества.\footnote{По идее данная теорема должна была быть в пункте 15.}}

\begin{theorem*}
Пусть $H$ - гильбертовое пространство, $M$ - замкнутое выпуклое в $H$. Тогда $\forall x \in H$ в $M$ $\exists !$(единственный) элемент $\hat{x} : 
\rho(x, M) = \norm{x - \hat{x}}$.
\end{theorem*}

\begin{proof}
\smallskip
\par\noindent \textbullet~Обозначим $d = \rho(x, M) = \inf_{y \in M} \norm{x - y}$. По определению точной нижней грани $\forall n \in \mathbb{N} \; \exists y_n \in M : 
d \le \norm{x - y_n} < d + \dfrac{1}{n}$.

\smallskip 
\noindent \textbullet~Рассмотрим точки $y_n, y_m, x - y_n, x - y_m$ и применим к ним равенство параллелограмма $2 \norm{x - y_n}^2 + 2 \norm{x - y_m}^2 = \norm{2x - y_n 
-y_m}^2 + \norm{y_n - y_m}^2$ - сумма квадратов диагоналей равна сумме квадратов его сторон.

\smallskip 
\noindent \textbullet~Из полученого выше $\norm{2x - y_n - y_m}^2 + \norm{y_n - y_m}^2 < 2(d + \dfrac{1}{n})^2 + 2(d + \dfrac{1}{m})^2$.

\smallskip
\noindent \textbullet~$\norm{2x - y_n - y_m}^2 = 4 \norm{x - \dfrac{y_n + y_m}{2}}^2$. $M$ - выпуклое, $y_n, y_m \in M \Rightarrow \dfrac{y_n + y_m}{2}$ - по выпуклости 
в $M$ будет лежать и их середина, то есть $\norm{x - \dfrac{y_n + y_m}{2}} \ge d$.

\smallskip
\noindent \textbullet~Подставляя в равенство, получаем $4d^2 + \norm{y_n - y_m}^2 < 2(d + \dfrac{1}{n})^2 + 2(d + \dfrac{1}{m})^2 = 4d^2 + \dfrac{4d}{n} + \dfrac{4d}{m} + 
\dfrac{2}{n^2} + \dfrac{2}{m^2} \Longleftrightarrow \norm{y_n - y_m}^2 < 2(d + \dfrac{1}{n})^2 + 2(d + \dfrac{1}{m})^2 = \dfrac{4d}{n} + \dfrac{4d}{m} + \dfrac{2}{n^2} +
\dfrac{2}{m^2}$. Выражение в правой части $\to 0$ при $n, m \to \infty$, а тогда и $\norm{y_n - y_m} \to 0$. По полноте $H$ $y_n, y_m \to y$. $M$ - замкнутое, $y_n \in M 
\Rightarrow y \in M$.

\noindent \textbullet~$d \le \norm{x - y_n} < d + \dfrac{1}{n}$, $y_n \to y \in M$. Устремляя $n \to \infty$ получаем, что $\norm{x - y} = d = \rho(x, M)$.

\noindent \textbullet~Докажем единственность. От противного: пусть нашлось $y^* \in M : \norm{x - y^*} = d$. Пишем равенство параллелограмма $2 \norm{x - y}^2 + 
2 \norm{x - y^*}^2 = \norm{2x - y - y^*}^2 + \norm{y - y^*}^2$. $\norm{x - y} = d$, $\norm{x - y^*} = d$, $\norm{2x - y - y^*} \ge 4 d^2$. Тогда получаем 
$4d^2 + \norm{y - y^*} \le 4d^2 \Rightarrow \norm{y - y^*} = 0 \Rightarrow y^* = y$. Найденный $y$ есть $\hat{x}$, который мы искали.
\end{proof}


\section{Координаты образа линейного оператора.}
\subsection*{Определение прямой суммы.}

\subsubsection*{Для 2-х попарно ортогональных подпростанств.}

\noindent\textbullet~В теории гильбертовых пространств важное значение имеет операция \textit{прямой суммы} попарно ортогональных подпространств. Пусть $H_1 \bot H_2$ 
в $H$, то есть $\forall x \in H_1, \forall y \in H_2 \Rightarrow (x, y) = 0 \; [x \bot y = 0]$. Полагаем $H_1 \oplus H_2 = \{ x_1 + x_2, x_1 \in H_1, x_2 \in H_2 \}$ 
- линейное многообразие в $H$. Проверим, что это множество замкнутое. Для этого берем последовательность точек $x_n \in H_1 \oplus H_2$, считаем, что $\exists x = \lim 
x_n$. Необходимо проверить, что $\lim x_n \Rightarrow x \in H_1 \oplus H_2$. 

\smallskip 
\noindent\textbullet~Так как $x_n \in H_1 \oplus H_2 \Rightarrow x_n = x_1^{(n)} +x_2^{(n)}$, где $x_1^{(n)} \in H_1, x_2^{(n)} \in H_2$, $x_1^{(n)} \bot x_2^{(n)}$.

\smallskip 
\noindent\textbullet~Составляем $x_n - x_m = (x_1^{(n)} - x_1^{(m)}) + (x_2^{(n)} - x_2^{(m)})$, тогда по теореме Пифагора $\norm{x_n - x_m}^2 = \norm{x_1^{(n)} - 
x_1^{(m)}}^2 + \norm{x_2^{(n)} - x_2^{(m)}}^2$. Левая часть $\to 0$, тогда и пара слагаемых $\norm{x_1^{(n)} - x_1^{(m)}}^2, \norm{x_2^{(n)} - x_2^{(m)}}^2$ $\to 0$. 
Тогда по полноте $H$ $\exists  x_1 = \lim x_1^{(n)}$, $\exists x_2 = \lim x_2^{(n)}$, причем $H_i$ - замкнуты, поскольку подпространства, тогда $x_i \in H_i$.

\smallskip 
\noindent\textasteriskcentered~Если вернуться к $x_n = x_1^{(n)} + x_2^{(n)}$, $x_n \to x$, $x_1^{(n)} \to x_1$, $x_2^{(n)} \to x_2$. Тогда $x = x_1 + x_2$, а в силу 
$x_i \in H_i \Rightarrow$ $x \in H_1 \oplus H_2$. Таким образом, это многообразие - замкнутое множество, а значит оно подпространство. Это позволяет определить прямую
сумму взаимноортогональных подпростанств, то есть линейное многообразие $H_1 \oplus H_2 = \{x_1 + x_2, x_j \in H_j \}$, которое является подпространством и называется
\textit{прямой суммой} $H_1$ с $H_2$.

\medskip
\noindent\textbullet~Далее в терминах прямой суммы если вернуться к основной теореме теории гильбертовых пространств, то тогда ясно, что эту теорему можно записать 
формулой $H = H_1 \oplus H_1^\bot$. Таким образом, любое гильбертовое пространство может быть разложено в прямую сумму $H_1$ и $H_1^\bot$.


\subsubsection*{Для n попарно ортогональных подпростанств.}

\noindent\textasteriskcentered~Если $H_1, \dots, H_p$ - подпространства и $i \neq j \; H_i \bot H_j$, то $H_1 \oplus \dots \oplus H_p = \{ x_1 + x_2 + \dots + x_p , 
x_i \in H_i \}$. Как выше устанавливается то, что это подпространство и называется \textit{прямой суммой} $H_1, \dots, H_p$ попарно ортогональных подпростанств.


\subsubsection*{Для последовательности попарно ортогональных подпростанств.}

\noindent\textbullet~Теперь перенесем эту операцию на целую последовательность $H_1, H_2, \dots$ попарно ортогональных подпространств. Рассматривать $x_1, x_2, \dots, x_j
\in H_j$ бессмысленно, потому что это ортогональный ряд, его сходимость равносильна сходимости $\sum_1^\infty \norm{x_j}^2$, а этот ряд оказывается расходящимся, потому 
что нормы не стремятся к нулю.

\smallskip 
\noindent\textasteriskcentered~Имея последовательность ортогональных подпространств $H_1, H_2, \dots$ создаем линейное многообразие $\hat{H} : 
\{ \sum_1^n x_k, x_k \in H_k\}$. После этого переходим к замыканию $Cl \hat{H}$ - подпространство $H$. Тогда это замыкание и обозначают $H_1 \oplus H_2 \oplus \dots \oplus \dots$ и называют \textit{ прямой суммой последовательности попарно ортогональных подпространств}. Таким образом обычно в функциональном анализе переносят операцию
с конечным числом слагаемых на операции с бесконечным числом слагаемых.


\subsubsection*{Математический смысл прямой суммы. }

\noindent\textbullet~В следующей теореме приводится без доказательства математический смысл прямой суммы.

\begin{theorem*}
Пусть $H_1, H_2, \dots$ - попарно ортогональные подпространства $H$, $\hat{H} = H_1 \oplus H_2 \oplus \dots$. Тогда $\forall x \in H$ его проекция на $\hat{H}$ $\hat{x}
= x_1 + x_2 + \dots$, где $x_n$ - проекция $x$ на $H_n$.
\end{theorem*}
\section{Действия с линейным операторами. Преобразование матрицы линейного оператора при переходе к новому базису.}
test
\section{Ядро и область значений линейного оператора.}
test
\section{Собственные векторы и собственные значения линейного оператора.}
\noindent\textasteriskcentered~Множество $S \subset C[a, b]$ называется равностепенно непрерывным семейством функций, если $\forall \epsilon > 0 \; \exists \delta > 0 : 
\abs{t'' - t'} \le \delta$, $t', t'' \in [a, b] \Rightarrow \abs{f(t'') - f(t')} \le \epsilon$ для $\forall f \in S$. 

\smallskip
\noindent\checkmark~Приведем без доказательства классическим критерий компактности в $C[a, b]$.

\begin{theorem*}[Арцела-Асколи]
Множество $K \subset C[a,b]$ - относительно компактно $\Longleftrightarrow$:

\smallskip 
\noindent 1)~$K$ - ограничено в $C[a, b]$;

\smallskip
\noindent 2)~$K$ - равностепенно непрерывно в $C[a, b]$.
\end{theorem*}
\section{Критерий представления матрицы в диагональном виде.}
\subsection*{Линейные оператор. Непрерывность. Ограниченность.}

\noindent\textasteriskcentered~Пусть $X, Y$ - НП, $A : X \rightarrow Y$ - отображение $X$ в $Y$, удовлетворяющее условию $A(\alpha x_1 + \beta x_2) = \alpha A x_1 + 
\beta A x_2$.  В этом случае говорят, что $A$ - \textit{линейный оператор}. Так как пространства нормированные, то можно говорить о непрерывности линейного оператора, 
что означает $x_n \to x \Rightarrow A_{x_n} \to A x$. Тогда говорят, что линейный оператор непрерывен в точке $x$. 

\smallskip
\noindent\textbullet~За счет линейности, если оператор непрерывен хотя бы в
одной точке, тогда он будет непрерывен и в любой точке. Пусть $A$ - \textit{непрерывен} в $x^*$, это означает, что если $x_n \to x^* \Rightarrow A x_n \to A x^*$. По арифметике
предела это означает, что $A x_n - A x^* \to 0$. По линейности оператора $A x_n - A x^* = A(x_n - x^*)$. Так как $x_n \to x^* \Rightarrow 
x_n - x^* \to 0$, тогда получается $A(x_n - x^*) \to 0$, что равно $A 0$ - значение оператора в нуле. Таким образом, из непрерывности в одной точке будет вытекать
непрерывность в нуле, а значит и в любой точке. 

\smallskip
\noindent\textbullet~Пояснение про нуль. Пусть $z_n \to 0 \Rightarrow A z_n \to A 0 = 0$. 
Поскольку $0 = \alpha 0$, тогда $A 0 = A(\alpha 0) = \alpha A(0)$. Тогда получится, что равенство $A(0) = \alpha A(0)$ верно при $\forall \alpha \Rightarrow A(0) = 0$.
Если знаем, что из $z_n \to 0 \Rightarrow A z_n \to 0$, тогда возьмем $\forall x \in X$, $x_n \to x \Rightarrow x_n - x \to 0$, тогда по непрерывности в нуле $A(x_n - x) \to 0$. По
линейности оператора $A(x_n - x) = A x_n - A x \to 0 \Longleftrightarrow A x_n \to Ax$ - значит линейный оператор непрерывен в точке $x$. Итого, линейный оператор
непрерывен в точке $x$ $\Longleftrightarrow$ оператор непрерывен в $0$.

\smallskip
\noindent\textasteriskcentered~Если $\exists M$ - $const > 0 : \forall x \in X \Rightarrow \norm{Ax} \le M \cdot \norm{x} \Rightarrow$ $A$ - \textit{ограниченный} оператор. 

\begin{theorem*}
    Линейный оператор $A$ - непрерывен $\Longleftrightarrow$ $A$ - ограничен
\end{theorem*}
\begin{proof}
\smallskip
\par\noindent\textbullet~Пусть оператор $A$ ограничен $\Rightarrow \norm{A x} \le M \cdot \norm{x}$. Если $x_n \to 0$, то написав неравенство $\norm{A x_n} \le 
M \cdot \norm{x_n}$ можно заметить, что правая часть $ \to 0$ $\Rightarrow \norm{A x_n} \to 0 \Rightarrow A x_n \to 0$. Таким образом, из ограниченности оператора 
вытекает непрерывность в нуле, а значит и в любой точке тоже.

\medskip
\noindent\textbullet~Пусть $A$ - непрерывен, допустим, что он не ограничен. Тогда $\forall n \in \mathbb{N}$ всегда $\exists x_n \in X : \norm{A x_n} > n \cdot 
\norm{x_n}$. Значит отсюда получится по линейности оператора и свойствам нормы $\norm{A \left( \dfrac{x_n}{n \cdot \norm{x_n}}\right)} > 1$. Если рассмотреть точки 
$y_n = \dfrac{x_n}{n \cdot \norm{x_n}}$, то очевидно окажется, что $\norm{y_n} = \dfrac{1}{n} \to 0$. Таким образом, $y_n \to 0 \Rightarrow$ по непрерывности 
$A y_n \to 0$, однако у нас выполняется неравенство $\norm{A y_n} > 1$, что противоречит тому, что $A y_n \to 0$. Значит оператор ограничен.
\end{proof}

\subsection*{Норма оператора. Аксиомы нормы.}

\noindent\textasteriskcentered~Для ограниченных операторов, если $\norm{x} \le 1$, то так как $\norm{A x} \le M \cdot \norm{x} \le M$, то тогда получаем, что $\sup \norm{A x}$ на 
единичном шаре конечен: $\sup_{\norm{x} \le 1} \norm{A x} < + \infty$. Эта величина называется \textit{нормой оператора} и обозначается $\norm{A}$. То есть $\norm{A} = 
\sup_{\norm{x} \le 1} \norm{A x}$. Если взять $\forall x \in X$, то точка $\dfrac{x}{\norm{x}}$ будет иметь единичную норму, то тогда по определению нормы оператора 
$\norm{A (\dfrac{x}{\norm{x}})} \le \norm{A}$, так как $\norm{x} \le 1$. С другой стороны $\norm{A (\dfrac{x}{\norm{x}})} = \dfrac{1}{\norm{x}} \cdot \norm{A x}$. А тогда получится, $\norm{A x} 
\le \norm{A} \cdot {\norm{x}} \; \forall x \in X$.

\medskip
\noindent\textbullet~Проверим, что $\norm{A}$ удовлетворяет всем аксиомам нормы на линейном многообразии. Рассмотрим линейное многообразие ограниченных операторов $V(X, Y)
= \{ A $ - линейный ограниченный оператор $: X \to Y\}$ ($V$ - знак линейной оболочки). Арифметические действия с операторами определяем поточечно: $(\alpha A)(x) = \alpha \cdot A(x)$, $(A + B)(x) = 
A x + B x$, при этом каждый раз будут получаться ограниченные операторы. Действительно если начать смотреть $\norm{(A + B) \cdot x} = \norm{A x + B x} \le \norm{A x} + 
\norm{B x}$, а так как операторы $A, B$ ограничены, то $\norm{A x} + \norm{B x} \le \norm{A} \norm{x} + \norm{B} \norm{x} = (\norm{A} + \norm{B}) \cdot \norm{x}$ = $const
\cdot \norm{x}$, значит оператор ограничен. В частности, если $\norm{x} \le 1 \Rightarrow \norm{(A + B) \cdot x} \le \norm{A} + \norm{B}$. Переходя к $\sup$ по $\norm{x}
\le 1 \Rightarrow \norm{A + B} \le \norm{A} + \norm{B}$ - доказали неравенство треугольника.

\medskip
\noindent\textbullet~Докажем аксиому $\norm{\alpha A} = \abs{\alpha}\norm{A}$. Считаем, что $\norm{x} \le 1$, вычисляем $\norm{(\alpha A)x} = \norm{\alpha A x} = 
\abs{\alpha} \cdot \norm{A x} \le \abs{\alpha} \cdot \norm{A}$. Таким образом, получили $\norm{\alpha A} \le \abs{\alpha} \cdot \norm{A}$.

\smallskip
\noindent\textbullet~Проверим противоположное неравенство. Запишем тождество $\norm{A} = \norm{\alpha (\dfrac{1}{\alpha} A)}$. В силу только что доказанного неравенства
подставляем $\dfrac{1}{\alpha}: \norm{\alpha (\dfrac{1}{\alpha} A)} \le \dfrac{1}{\abs{\alpha}} \cdot \norm{\alpha A} \Rightarrow \abs{\alpha} \cdot \norm{A} \le  
\norm{\alpha A}$. Получили противоположное неравенство, значит $\abs{\alpha} \cdot \norm{A} = \norm{\alpha A}$. Проверили вторую аксиому. Первая аксиома очевидна. 

\medskip 
\noindent\textbullet~Теперь можно вести разговор о линейном многообразии ограниченных оператор. В этом многообразии величина $\norm{A} = \sup_{\norm{x} \le 1} \norm{A x}$
задает норму. И значит наше линейное многообразие превращается в линейное нормированное пространство. Значит мы можем говорить об операторе $A = \lim A_n$, понимая под 
этим тот факт, что $\norm{A_n - A} \to 0$, то есть $\forall \epsilon > 0 \; \exists N : \forall n \ge N \Rightarrow \norm{A_n - A} \le \epsilon$. Норма разности - 
$\sup_{\norm{x} \le 1} \norm{A_n x - A x}$, а тогда получается, что $\norm{A_n - A} \to 0$ можно перезаписать в форме $\forall \epsilon > 0 \; \exists N : \forall n \ge 
N$ и $\forall x : \norm{x} \le 1 \Rightarrow \norm{A_n x - A x} \le \epsilon$. Последнее поточеное неравенство должно выполняться сразу для всех иксов для единичного 
шара, начиная с какого-то номера.

\section{Евклидово пространство. Определение. Неравенство Коши-Буняковского. Неравенство треугольника.}
test
\section{Дифференциальные уравнения первого порядка. Понятия уравнения и его решения. Поле направлений. Задача Коши. Теорема Пикара. Общее, частное и особое решение.}
test
\section{Методы интегрирования уравнений первого порядка. Уравнения с разделяющимися переменными. Однородные уравнения и уравнения, приводящиеся к однородным.}
\noindent\textbullet~Далее, не оговаривая, мы считаем, что все пространства являются B-пространствами, то есть $x_n - x_m \to 0 \Rightarrow \exists \lim x_n$.
Обратное верно всегда.

\smallskip 
\noindent\textbullet~Для доказательства потребуется принцип вложенных шаров: $X$ - B-пространство, $\overline{V}_n$ - замкнутые шары в $X$, $\overline{V}_{n+1} \subset 
\overline{V}_n \Rightarrow \bigcap_{n = 1}^\infty \overline{V}_n \neq \emptyset$, общих точек не обязательно должна быть единственной. 
Если $r \to 0$, то общих точек только 1. Например, $\overline{V}_n = [-1 - \frac{1}{n}, 1 + \frac{1}{n}]$.

\begin{theorem*}[Банах, Штейнгауз]
Пусть дана последовательность линейных ограниченных операторов $A_n \in V(X, Y)$, про которую известно, что $\forall x \in X$ $\sup_{n \in \mathbb{N}}\norm{A_n x} < 
+ \infty$ (то есть последовательность операторов поточечно равномерно ограничена). Тогда $\sup_{n \in \mathbb{N}} \norm{A_n} < + \infty$ (то есть последовательность 
операторов просто равномерно ограничена). Равномерность означает конечность соответствующих супремумов.
\end{theorem*}

\begin{proof}
\smallskip 
\par\noindent\textbullet~Доказательство разобьем на 2 этапа.

\smallskip
\noindent\textbullet~Допустим $\exists \overline{V} = \overline{V}_r(a) : \sup_{x \in \overline{V}, n \in \mathbb{N}} \norm{A_n x} < +\infty$. Покажем тогда, что можно 
утверждать, что из этого факта будет вытекать, что $sup_{n \in \mathbb{N}} \norm{A_n} < +\infty$. Обозначим для удобства $M = \sup_{x \in \overline{V}, n \in \mathbb{N}}
\norm{A_n x}$. 

\smallskip
\noindent\textbullet~Рассмотрим единичный шар $\overline{V}_1 = \overline{V}_1(0)$. По нему считаются нормы операторов. Возьмем $\forall x \in \overline{V}_1$ и определяем 
$y = a + r x$. Если составить норму разности $\norm{y - a} = \norm{r x} = r \norm{x} \le r$. Таким образом, точка $y \in \overline{V}$. Значит $\forall n \in \mathbb{N}
\Rightarrow \norm{A_n y} \le M$. Из формулы $y = a + rx \Rightarrow x = \dfrac{y - a}{r}$, начинаем смотреть, что представляет норма значения $n$-ого оператора над точкой 
$x$, которая является любой.

\smallskip
\noindent\textbullet~$\norm{A_n x} = \norm{a_n (\dfrac{y - a}{r})} = \dfrac{1}{r} \norm{A_n y - A_n a} \le \dfrac{1}{r}(\norm{A_n y} + \norm{A_n a}) \le \dfrac{1}{r} 
(M + \norm{A_n a})$. Ясно, что норма $\norm{A_n a} \le \sup_{m \in \mathbb{N}} \norm{A_m a} = N < + \infty$ по условию теоремы (поточечно равномерно ограничена). 
Подставляя это в последнее неравенство $\norm{A_n x} \le \dfrac{1}{r} (M + N)$ - не зависит от $n$ и $x \in \overline{V}_1$. Тогда сначала переходим к $\sup$ по $x \in 
\overline{V}_1$, а тогда получаем, что $\norm{A_n} \le \dfrac{1}{r}(M + N)$. Теперь переходим к супремому по номерам $n \Rightarrow \sup_{n \in N} \norm{A_n} < + \infty$,
то есть будет выполняться утверждение теоремы Банаха-Штейнгауза. Первый этап проделали.

\medskip
\noindent\textbullet~Допустим, что $\nexists$ шара $\overline{V}$ из первого этапа и убедимся в том, что тогда появится противоречие 
(такой шар хотя бы один должен существовать). Возьмем $\forall \overline{V}$, тогда по предположению должна $\exists x_1 \in V, \exists n_1 \in \mathbb{N} : 
\norm{A_{n_1}(x_1)} > 1$. 

\smallskip
\noindent\textbullet~Оператор $A_{n_1}$ непрерывен (поскольку ограничен), тогда по стандартному свойству непрерывности $\exists \overline{V}_{r_1}(x_1) = \overline{V}_1 : 
\overline{V}_1 \subset \overline{V}, \forall y \in \overline{V}_1 \Rightarrow \norm{A_{n_1}(y)} > 1$. Построенный шар $\overline{V}_1$ не может быть шаром из первого 
этапа по нашему предположению, тогда $\exists x_2 \in V_1, \exists n_2 \in \mathbb{N} : \norm{A_{n_2}(x_2)} > 2$, при этом можно считать, что $n_2 > n_1$. Тогда опять по 
непрерывности $\exists \overline{V}_{r_2} (x_1) = \overline{V}_2 : \overline{V}_2 \subset V_1$, $r_2 < \frac{r_1}{2}$, $\forall y \in \overline{V}_2 \Rightarrow 
\norm{A_{n_2}(y)} > 2$ и так далее продолжаем это построение.

\smallskip
\noindent\textbullet~В результате выстраивается последовательность замкнутые вложенных шаров: $\overline{V}_{k+1} \in \overline{V}_k$, $r_k \to 0$, радиус каждый раз 
уменьшается в 2 раза, и при этом $\forall x \in \overline{V}_k \Rightarrow \norm{A_{n_k}(x)} > k$. По принципу вложенных шаров существует точка $x^* \in \bigcap_{k = 1}^\infty \overline{V}_k$. В частности эта точка принадлежит шару $\overline{V}_k$, а тогда $\norm{A_{n_k}(x^*)} > k$. Если в этом неравенстве $k \to \infty \Rightarrow
\norm{A_{n_k}(x^*)} \to +\infty$. А это противоречит тому, что $\sup_{m \in \mathbb{N}}\norm{A_n(x^*)} < +\infty$. Полученное противоречие доказывает, что шар из первого 
этапа существует, значит теорема доказана.
\end{proof}


\subsection*{Следствие из теоремы. Интерпретации $A  = \lim A_n$.}

\noindent\textbullet~Пусть $A_n \in V(X, Y), A = \lim A_n$. В функциональном анализе есть 3 разных понимания этого равенства.

\smallskip 
1) $\norm{A_n - A} \to 0$ - оператор $A$ является пределом по операторной норме. Это тоже самое, что $\forall \epsilon > 0 \exists n_0 : \forall n \ge n_0 \Rightarrow
\norm{A_n x - A_x} \le \epsilon$ сразу для всех $x$ из замкнутого единичного шара. - равномерная сходимость.

\smallskip 
2) $\forall x \in X \Rightarrow A x = \lim A_n x$ - сильная (поточечная) сходимость последовательности операторов.

\smallskip 
3) $\forall f$ - линейного ограниченого функционала, $\forall x \Rightarrow f(A x) = \lim f(A_n x)$ - слабая сходимость последовательности операторов.

\bigskip\noindent\textbf{Следствие.}\textit{ Пусть $A_n \in V(X, Y)$, про которую известно, что $\forall x \in X \Rightarrow \exists \lim A_n x = A x$. Тогда предельный 
оператор $A \in V(X, Y)$, то есть тоже ограничен (по сильному пределу). }

\begin{proof}
\smallskip
\par\noindent\textbullet~Возьмем $x \in \overline{V}_1 \Rightarrow \norm{A x} \le M$ - $const$. $\norm{A x} = \norm{(A x - A_n x) + A_n x} \le \norm{A x - A_n x} + 
\norm{A_n x}$. Для имеющегося $x$, так как можно написать $A x = \lim A_n x$, возьмем $\epsilon = 1, \; \exists n_0 : \forall n \ge n_0 \Rightarrow \norm{A x - A_n x} \le 
1$. В частности, получится $\norm{A x} \le 1 + \norm{A_{n_0} x}$. Норма $\norm{A_{n_0}x} \le \norm{A_{n_0}} \cdot \norm{x} \le \norm{A_{n_0}}$.

\smallskip
\noindent\textbullet~Так как $\forall x \; \exists \lim A_n x$ по условию следствия, тогда по стандартным свойствам предела $\{ \norm{A_n x}\}$ - ограничена. То есть для 
любого $x$ выполняется условия Банаха-Штейнгауза, а тогда $S = \sup \norm{A_n} < +\infty$. Тогда возвращаясь к неравенству $\norm{A x} \le 1 + \norm{A_{n_0} x}$ получаем, 
что $\norm{A x} \le 1 + \norm{A_{n_0}} \le 1 + S$ - $const$. А следовательно неравенство верно $\forall x \in \overline{V}_1 \Rightarrow \norm{A} < +\infty$.
\end{proof}
\section{Линейные уравнения первого порядка. Уравнения Бернулли.}
text
\section{Уравнения в полных дифференциалах. Интегрирующий множитель.}
text
\section{Уравнения первого порядка, не разрешенные относительно производной. Уравнения Лагранжа и Клеро.}
text
\section{Дифференциальные уравнения высших порядков. Основные понятия и определения. Задача Коши. Теорема Пикара. Понижение порядка уравнения. Уравнения, не содержащие искомой функции и последовательных первых производных. Уравнения, не содержащие независимой переменной.}
text
\section{Линейные дифференциальные уравнения n-го порядка. Свойства решений линейного однородного уравнения. Фундаментальная система решений и определитель Вронского. Признак линейной независимости решений. Формула Остроградского–Лиувилля.}
text
\section{Построение общего решения линейного однородного уравнения по фундаментальной системе решений. Структура общего решения неоднородного уравнения. Принцип наложения. Метод вариации произвольных постоянных (метод Лагранжа) для уравнения 2-го порядка. Случай уравнения  n-го порядка.}
text
\section{Системы дифференциальных уравнений. Основные понятия и определения. Нормальная система. Задача Коши. Механическое истолкование нормальной системы и ее решения. Теорема Пикара. Связь между уравнениями высшего порядка и системами дифференциальных уравнений 1-го порядка.}
text
\section{Линейные системы. Свойства линейных систем. Фундаментальная матрица. Определитель Вронского. Критерий линейной независимости вектор-функций. Формула Остроградского–Лиувилля.}
text
\section{Построение общего решения линейной однородной системы по фундаментальной системе решений. Интегрирование линейной однородной системы с постоянными коэффициентами методом Эйлера.}
text
\section{Структура общего решения неоднородной линейной системы. Метод вариации произвольных постоянных (метод Лагранжа).}
text
\section{Числовые ряды. Сходимость. Необходимый признак сходимости.}
text
\section{Свойства сходящихся рядов.}
text
\section{Признаки сравнения рядов с положительными членами.}
test
\section{Признаки Даламбера, Коши и интегральный сходимости рядов.}
test
\section{Знакочередующиеся ряды. Теорема Лейбница.}
test
\section{Знакопеременные ряды. Абсолютная и условная сходимость. Свойства абсолютно сходящихся рядов.}
test
\section{Функциональные ряды. Область сходимости. Мажорируемые ряды. Равномерная сходимость. Признак Вейерштрасса.}
test
\section{Непрерывность суммы ряда.}
test
\section{Интегрирование функционального ряда.}
test
\section{Дифференцирование функционального ряда.}
test
\section{Степенные ряды. Теорема Абеля. Интервал сходимости.}
test
\section{Радиус сходимости. Формулы Даламбера и Коши–Адамара.}
test
\section{Непрерывность суммы степенного ряда. Интегрирование и дифференцирование степенного ряда.}
test
\section{Ряды Тейлора и Маклорена. Критерий сходимости ряда Тейлора.}
test
\section{Тригонометрическая система функций. Ряд Фурье. Разложение периодической функции в ряд Фурье. Теорема Дирихле.}
test
\section{Ряды Фурье для чётных и нечётных функций. Ряд Фурье непериодической функции.}
test
\section{Ряд Фурье с комплексными членами.}
test

\end{sloppypar}
\end{document}