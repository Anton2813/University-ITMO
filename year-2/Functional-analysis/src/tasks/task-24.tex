\subsection*{Теорема о продолжении по непрерывности.}

\noindent\textbullet~Пусть $Z$ - линейное многообразие в $X$. Также пусть имеется $Y$ (это все НП), а также оператор $A : Z \to Y$. Раз $Z$ - линейное многообразие, то будем предполагать, что $A$ - ограничен на $Z$. Норма $\norm{A}$ на $Z$, $\norm{A} = \sup_{x \in Z} \norm{A x}, \norm{x} \le 1$. Возникает 
вопрос, при каких условиях на $Z$ и $Y$ $\exists \hat{A} \in V(X, Y) : $

1)$\hat{A} \big|_z = A$;

2)$\norm{\hat{A}} = \norm{A}$. 

\noindent\textasteriskcentered~Тогда говорят, что оператор $\hat{A}$ является \textit{продолжением ограниченного оператора $A$ по непрерывности}. 

\medskip
\begin{theorem*}
Пусть $Z$ всюдо плотно в $X$, $Y$ - Банахово пространство. Тогда оператор $\hat{A}$ существует и только один. 
\end{theorem*}

\begin{proof}
\smallskip
\par\noindent\textbullet~По условию $Cl Z = X$, это означает, что $\forall x \in X \; \exists z_n \in Z : x = \lim z_n$ в $X$. Рассмотрим последовательность значений 
оператора $A : Z \to Y$ на точках $z_n$. По условию оператор $A$ - ограничен на $Z$, то есть $\norm{A} = \sup_{z \in Z} \norm{A z} < + \infty, \norm{z} \le 1$.

\smallskip
\noindent\textbullet~Рассмотрим $\norm{A z_m - A z_n} = \norm{A (z_m - z_n)} \le \norm{A} \cdot \norm{z_m - z_n}$. Норма $\norm{z_m - z_n} \to 0$, так как $z_m \to x$, а 
тогда из написанного неравенства $\norm{A z_m - A z_n} \to 0 \Rightarrow A z_m $ сходится в себе в пространстве $Y$, которое полное. Значит у этой последовательности 
существует предел, обозначим его $\hat{A}x$. 

\smallskip
\noindent\textbullet~Проверим, что наше определение является корретным в том смысле, если помимо $z_m \to x$ найдется последовательность $z_m' \to 
x$, то тогда $\lim A z_m = \lim A z_m'$. Для того, чтобы это проверить составим $\norm{A z_m - A z_m'} = \norm{A (z_m - z_m')} \le \norm{A} \norm{z_m - z_m'}$, где 
$\norm{z_m - z_m'} \to 0$, а тогда $\norm{A z_m - A z_m'} \to \Rightarrow \lim A z_m = \lim A z_m'$.

\smallskip
\noindent\textbullet~Итак мы проверили, что формула $z_m \in Z, z_m \to x, \hat{A} x = \lim A z_m$ корректно определяет некоторый оператор $\hat{A}$, заданный на всем $X$.
По арифметике предела последовательностей ясно, что оператор $\hat{A}$ является линейным. Если при этом в этой формуле $x \in Z$, то последовательность $z_m = x \to x$, 
а тогда получается, что значение оператора $\hat{A} x = A x$, то есть этот линейный оператор является продолжением линейного оператора $A$ со всюду плотного линейного 
многообразия $Z$ на все $X$. Осталось проверить, что оператор $\hat{A}$ ограничен и его норма $\norm{\hat{A}} = \sup_{x \in X, \norm{x} \le 1} \norm{\hat{A} x} 
= \norm{A}$.

\medskip
\noindent\textbullet~Доказывая ограниченность, воспользуемся формулой, с помощью которой мы продолжали оператор. $\forall x \in X, z_m \to x, z_m \in Z$, $A$ - ограничен.
Тогда можем писать $\norm{A z_m} \le \norm{A} \cdot \norm{z_m}$, где $\norm{z_m} \to \norm{x}$, $\norm{A z_m} \to \norm{\hat{A}x}$, в пределе это неравенство сохранится, 
тогда получится неравенство $\norm{\hat{A} x} \le \norm{A} \cdot \norm{x} \Rightarrow \hat{A}$ - ограничен и подставляя в это неравенство $\norm{x} \le 1$ получаем 
$\norm{\hat{A} x } \le \norm{A} \Rightarrow \norm{\hat{A}} \le \norm{A}$. Противоположное неравенство $\norm{\hat{A}} \ge \norm{A}$ - очевидно, так как оператор $\hat{A}$ 
является продолжение оператора $A$ на все $X$ ($\hat{A}\big|_z = A$). Таким образом, мы проверили, что построенный оператор ограничен и его норма совпадает с нормой 
исходного оператора. 

\medskip 
\noindent\textbullet~Если бы существовал другой оператор $\tilde{A}$ с такими же свойствами, то есть 1) $\tilde{A}\big|_z = A$, 2) $\norm{\tilde{A}} = \norm{A}$, то 
тогда бы получилось, что мы опять берем $\forall z \in X, \; \exists z_m \to z, z_m \in Z$. Тогда оба продолженных оператора $\hat{A}, \tilde{A}$ были бы непрерывными, 
поэтому $\hat{A} z_m \to \hat{A} x$, $\tilde{A} z_m \to \tilde{A} x$ по непрерывности. Но так как оба этих оператора продолжают оператор $A$ с многообразия $Z$, то точки 
$\tilde{A}z_m = \hat{A} z_m = A z_m$, а тогда в этих двух предельных соотношения слева можно подставлять $A z_m \to \hat{A} x, \tilde{A} x$, а тогда по единственности 
предела точки $\hat{A} x = \tilde{a} x$ совпадут, что верно для всех $x$, а значит это два одинаковых оператора.
\end{proof}


\subsection*{Иллюстрация неограниченного оператора. Операторы в пространствах $l_p$, $C[0, 1]$.}

\noindent\checkmark~Убедимся, что есть линейные неограниченные операторы. Для этого рассмотрим в качестве $X = \{$непрерывно дифф на $[0, 1] f, \norm{f} = \max_{[0, 1]}
\abs{f}\}$. Ясно, что это линейное многообразие. В качестве $Y$ возьмем $C[0,1] = \{$ непр $[0, 1] f$ с sup-нормой, определенной выше $\}$. Рассмотрим оператор $A : X \to 
Y$, $A(f) = f'$, этот оператор линейный по правилам дифференциирования. Если бы он был ограничен, то $\norm{A(f)} \le M \cdot \norm{f}$, $M$ - $const$. Или, подставлял 
значение оператора, что $\norm{f'} \le M \cdot \norm{f}$. Очевидно, что это невозможно, потому что можно взять асцилирующую фунцию, соответственно значение производной 
у нее может быть сколь угодно большой. Соотетсвенно, этот линейный оператор не ограничен. 

\smallskip
\noindent\checkmark~Мы определили $\norm{A} = \sup_{\norm{x} \le 1} \norm{A x}$, но следует понимать, что в конкретных ситуациях само вычисление нормы оператора может 
оказаться неподъемной задачей, поэтому в большинстве приложений можно судить о норме оператора только записывая оценки этой нормы, потому что точное значение нормы не 
сосчитать.

\bigskip 
\noindent\textbullet~В пространствах $l_p$ операторы возникают на основе стандартных базисных точек $\overline{e}_n = \{ 0, 0, \dots, 0, 1, 0, \dots \}$. Например, 
рассмотрим $\lambda_j \in \mathbb{R} : \abs{\lambda_j} \le M$. Возьмем $\forall \overline{x} = (x_1, x_2, \dots) \in l_p$ и рассмотрим формальный ряд $\sum_1^\infty 
\lambda_n x_n \overline{e}_n$. В НП должны смотреть, что частичные суммы ряда $S_n = \sum_{k = 1}^\infty \lambda_k x_k \overline{e}_k$ и определить есть ли у него
предел или нет, тогда $S_n = (\lambda_1 x_1, \lambda_2 x_2, \dots, \lambda_n x_n, 0, \dots)$. 

\medskip
\noindent\textbullet~Если теперь рассмотреть точку $S = (\lambda_1 x_1, \lambda_2 x_2, \dots, \lambda_n x_n , \dots)$ и составить сумму $\sum_1^\infty 
\abs{\lambda_n x_n}^p \le M^p \le \sum_1^\infty \abs{x_n}^p < +\infty$, так как $\sum_1^\infty \abs{x_n}^p < +\infty$. Тогда эта последовательность $S \in l_p$. Если 
теперь рассмотреть $\norm{S - S_n}_p = \norm{(0, \dots, 0, \lambda_{n+1} x_{n + 1}, \dots)}_p = (\sum_{k = n + 1}^\infty \abs{\lambda_k}^p \abs{x_k}^p)^\frac{1}{p} \le 
M (\sum_{k = n+ 1}^\infty \abs{x_k}^p)^\frac{1}{p}$, где $\sum_{k = n + 1}^\infty \abs{x_k}^p \to 0$ по условию вхождения в пространство $l_p$, так как 
$\sum_{k = 1}^\infty \abs{x_k}^p < +\infty$. Тогда $S = \lim S_n$. Таким образом, $A\overline{x} = \sum_{n = 1}^\infty \lambda_n x_n \overline{e}_n, \abs{\lambda_n} \le M$ 
определеляет линейный оператор $l_p \to l_p$. Предыдущие вычисления показывают, что $\norm{A \overline{x}}_p \le M \norm{\overline{x}}_p \Rightarrow \norm{A} \le M < 
+\infty$.

\medskip
\noindent\textbullet~Теперь рассмотрим $C[0, 1]$, $K(u,v)$ - непрерывная фукнция двух переменных на $[0,1] \times [0, 1]$, $M = \max_{[0, 1]^2} \abs{K(n, v)} < +\infty$, 
это значение конечно по т.~Вейерштрасса о непрерывных функциях. Теперь возьмем любую функцию $f \in C[0, 1]$, $A(f, x) = \int_0^1 K(x, t) f(t) dt$, определяющая некоторую 
функцию переменной $x$, она непрерывна по свойствам интеграла и является линейным оператором $C[0, 1] \to [0, 1]$. Также по свойствам интеграла $\abs{A(f, x)} \le 
\int_0^1 \abs{K(x, t)} \abs{f(t)} dt \le \int_0^1 M \norm{f} dt = M \cdot \norm{f}$. Написанный оператор $\norm{A f} \le M \norm{f}, \norm{A}$ ограничен и его 
норма  $\le M$. В функциональном анализе такой класс обозначается как операторы Фредгольма.
