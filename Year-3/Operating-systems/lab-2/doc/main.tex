%форматирование размера документа
\documentclass[11pt, a4paper]{article}

\usepackage{geometry}
% total - determines printable width, height
\geometry{ 
	a4paper, total={180mm,267mm}
}

%----text,fonts------------------------------------------------------------------------------------
% \usepackage{fontspec}
\usepackage{mmap}
\usepackage[T2A]{fontenc}
\usepackage[utf8]{inputenc}
\usepackage[english, russian]{babel}
\usepackage{setspace}
\setstretch{0,9}
\usepackage{fancyvrb}
\usepackage{courier}
% \setmonofont{FiraCode Nerd Font Mono}

%----math,graphics---------------------------------------------------------------------------------
\usepackage{amsmath,amsfonts,amssymb}
\usepackage{amsthm}
\usepackage{listings}
\usepackage{xcolor}
\usepackage{filecontents}
\usepackage{fancyvrb}
\usepackage{fvextra}
\usepackage{hyperref}
\hypersetup{
    colorlinks=true,
    linkcolor=blue,
    filecolor=magenta,      
    urlcolor=blue,
    pdftitle={Overleaf Example},
    pdfpagemode=FullScreen,
}
\VerbatimFootnotes % Required, otherwise verbatim does not work in footnotes!
\usepackage{tikz}
\usetikzlibrary{calc}
\usepackage{pgfplots}
\pgfplotsset{
	compat=1.17
}

\usepackage{graphicx}
\graphicspath{{fig/}}
  
\usepackage{wrapfig}
\usepackage{tabularx}

% relative importing
\usepackage{import}

% --- ugly internals for language definition ---
%
\makeatletter

% initialisation of user macros
\newcommand\PrologPredicateStyle{}
\newcommand\PrologVarStyle{}
\newcommand\PrologAnonymVarStyle{}
\newcommand\PrologAtomStyle{}
\newcommand\PrologOtherStyle{}
\newcommand\PrologCommentStyle{}

% useful switches (to keep track of context)
\newif\ifpredicate@prolog@
\newif\ifwithinparens@prolog@

% save definition of underscore for test
\lst@SaveOutputDef{`_}\underscore@prolog

% local variables
\newcount\currentchar@prolog

\newcommand\@testChar@prolog%
{%
  % if we're in processing mode...
  \ifnum\lst@mode=\lst@Pmode%
    \detectTypeAndHighlight@prolog%
  \else
    % ... or within parentheses
    \ifwithinparens@prolog@%
      \detectTypeAndHighlight@prolog%
    \fi
  \fi
  % Some housekeeping...
  \global\predicate@prolog@false%
}

% helper macros
\newcommand\detectTypeAndHighlight@prolog
{%
  % First, assume that we have an atom.
  \def\lst@thestyle{\PrologAtomStyle}%
  % Test whether we have a predicate and modify the style accordingly.
  \ifpredicate@prolog@%
    \def\lst@thestyle{\PrologPredicateStyle}%
  \else
    % Test whether we have a predicate and modify the style accordingly.
    \expandafter\splitfirstchar@prolog\expandafter{\the\lst@token}%
    % Check whether the identifier starts by an underscore.
    \expandafter\ifx\@testChar@prolog\underscore@prolog%
      % Check whether the identifier is '_' (anonymous variable)
      \ifnum\lst@length=1%
        \let\lst@thestyle\PrologAnonymVarStyle%
      \else
        \let\lst@thestyle\PrologVarStyle%
      \fi
    \else
      % Check whether the identifier starts by a capital letter.
      \currentchar@prolog=65
      \loop
        \expandafter\ifnum\expandafter`\@testChar@prolog=\currentchar@prolog%
          \let\lst@thestyle\PrologVarStyle%
          \let\iterate\relax
        \fi
        \advance \currentchar@prolog by 1
        \unless\ifnum\currentchar@prolog>90
      \repeat
    \fi
  \fi
}
\newcommand\splitfirstchar@prolog{}
\def\splitfirstchar@prolog#1{\@splitfirstchar@prolog#1\relax}
\newcommand\@splitfirstchar@prolog{}
\def\@splitfirstchar@prolog#1#2\relax{\def\@testChar@prolog{#1}}

% helper macro for () delimiters
\def\beginlstdelim#1#2%
{%
  \def\endlstdelim{\PrologOtherStyle #2\egroup}%
  {\PrologOtherStyle #1}%
  \global\predicate@prolog@false%
  \withinparens@prolog@true%
  \bgroup\aftergroup\endlstdelim%
}

% language name
\newcommand\lang@prolog{Prolog-pretty}
% ``normalised'' language name
\expandafter\lst@NormedDef\expandafter\normlang@prolog%
  \expandafter{\lang@prolog}

% language definition
\expandafter\expandafter\expandafter\lstdefinelanguage\expandafter%
{\lang@prolog}
{%
  language            = Prolog,
  keywords            = {},      % reset all preset keywords
  showstringspaces    = false,
  alsoletter          = (,
  alsoother           = @$,
  moredelim           = **[is][\beginlstdelim{(}{)}]{(}{)},
  MoreSelectCharTable =
    \lst@DefSaveDef{`(}\opparen@prolog{\global\predicate@prolog@true\opparen@prolog},
}

% Hooking into listings to test each ``identifier''
\newcommand\@ddedToOutput@prolog\relax
\lst@AddToHook{Output}{\@ddedToOutput@prolog}

\lst@AddToHook{PreInit}
{%
  \ifx\lst@language\normlang@prolog%
    \let\@ddedToOutput@prolog\@testChar@prolog%
  \fi
}

\lst@AddToHook{DeInit}{\renewcommand\@ddedToOutput@prolog{}}

\makeatother
%
% --- end of ugly internals ---


% --- definition of a custom style similar to that of Pygments ---
% custom colors
\definecolor{PrologPredicate}{RGB}{000,031,255}
\definecolor{PrologVar}      {RGB}{024,021,125}
\definecolor{PrologAnonymVar}{RGB}{000,127,000}
\definecolor{PrologAtom}     {RGB}{186,032,032}
\definecolor{PrologComment}  {RGB}{063,128,127}
\definecolor{PrologOther}    {RGB}{000,000,000}

% redefinition of user macros for Prolog style
\renewcommand\PrologPredicateStyle{\color{PrologPredicate}}
\renewcommand\PrologVarStyle{\color{PrologVar}}
\renewcommand\PrologAnonymVarStyle{\color{PrologAnonymVar}}
\renewcommand\PrologAtomStyle{\color{PrologAtom}}
\renewcommand\PrologCommentStyle{\itshape\color{PrologComment}}
\renewcommand\PrologOtherStyle{\color{PrologOther}}

% custom style definition 
\lstdefinestyle{Prolog-pygsty}
{
  language     = Prolog-pretty,
  upquote      = true,
  stringstyle  = \PrologAtomStyle,
  commentstyle = \PrologCommentStyle,
  literate     =
    {:-}{{\PrologOtherStyle :-}}2
    {,}{{\PrologOtherStyle ,}}1
    {.}{{\PrologOtherStyle .}}1
}

% global settings
\lstset
{
  captionpos = below,
  frame      = single,
  columns    = fullflexible,
  basicstyle = \ttfamily,
}


\begin{document}

\import{.}{titular.tex}

\newpage

\section{Текст задания}
Разработать комплекс программ на пользовательском уровне и уровне ядра, который собирает информацию на стороне ядра и передает информацию на уровень пользователя, и выводит ее в удобном для чтения человеком виде. Программа на уровне пользователя получает на вход аргумент(ы) командной строки (не адрес!), позволяющие идентифицировать из системных таблиц необходимый путь до целевой структуры, осуществляет передачу на уровень ядра, получает информацию из данной структуры и распечатывает структуру в стандартный вывод. Загружаемый модуль ядра принимает запрос через указанный в задании интерфейс, определяет путь до целевой структуры по переданному запросу и возвращает результат на уровень пользователя.

\bigskip
\noindent
Интерфейс передачи между программой пользователя и ядром и целевая структура задается преподавателем. Интерфейс передачи:

\begin{itemize}
  \item ioctl - передача параметров через управляющий вызов к файлу/устройству.
\end{itemize}

\section{Выполнение}

\subsection{Листинг модуля}

\lstinputlisting[
  style = User-C,
caption = {\texttt{\detokenize{ioctl.h}}}
  ]{../mod/ioctl.h}

\lstinputlisting[
  style = User-C,
  caption ={\texttt{\detokenize{ioctl_dev.h}}} 
  ]{../mod/ioctl_dev.h}

\lstinputlisting[
  style = User-C,
  caption ={\texttt{\detokenize{ioctlmod.c}}} 
]{../mod/ioctlmod.c}


\subsection{Листинг пользовательской программы}

\lstinputlisting[
  style = User-C,
  caption ={\texttt{\detokenize{main.c}}} 
]{../app/main.c}

\subsection{Скрипты}

\lstinputlisting[
  style = User-scripts,
  caption ={\texttt{\detokenize{Makefile}} модуля} 
]{../mod/Makefile}

\lstinputlisting[
  style = User-scripts,
  caption ={\texttt{\detokenize{build-archive.sh}}} 
]{../scripts/build-archive.sh}

\lstinputlisting[
  style = User-scripts,
  caption ={\texttt{\detokenize{test-module.sh}}} 
]{../scripts/test-module.sh}

\lstinputlisting[
  style = User-scripts,
  caption ={\texttt{\detokenize{install-kernel.sh}}} 
]{../scripts/install-kernel.sh}

\subsection{Результаты работы программы}

\begin{lstlisting}[
  style = User-output,
  caption = {Output}
]
nikit@vmk:~/data/linux$ make -C lab/app run
make: Entering directory '/home/nikit/data/linux/lab/app'
sudo ./build/app
[INFO]: Open device
IOCTL_TEST: received value(HEX)=12345678
IOCTL_READ_MEMBLOCK: received value struct user_memblock=0x7ffd1b2b1890: {
  bottom_up=0
  current_limit=4831838208
  memory=0xffffffffaba70d90
  reserved=0xffffffffaba70db8
}
IOCTL_READ_PCIDEV: received value struct user_pci_dev=0x7ffd1b2b18b8: {
  devfn=0
  vendor=32902
  device=4663
  subsystem_vendor=0
  subsystem_device=0
}
[INFO]: Close device
make: Leaving directory '/home/nikit/data/linux/lab/app'

nikit@vmk:~/data/linux$ dmesg
[   66.059766] ioctlmod: loading out-of-tree module taints kernel.
[   66.060171] ioctl_mod: interface loaded
[   66.063085] ioctl_mod: dev=dev_ioctl, class=devc_ioctl created
[  126.195155] ioctl_mod: OPENED
[  126.195163] ioctl_mod: ioctl(IOCTL_TEST)
[  126.195164]  received=2271560481
[  126.195166]  sent=305419896
[  126.195950] ioctl_mod: ioctl(IOCTL_READ_MEMBLOCK)
[  126.197450] ioctl_mod: ioctl(IOCTL_READ_PCIDEV)
\end{lstlisting}

\section{Вывод}
В ходе выполнения данной лабораторной работы были выполнены исследования в области того, каким образом можно экспортировать функции ядра, если они изначально недоступны для модулей. Также было понято, что драйверы - тоже модули ядра, соответственно по заданию требовалось написать свой драйвер.
Кроме того, было изучено, как писать свой драйвер ioctl, syscall, собирать ядро и модули с основной машины для виртуальной для более быстрой компиляции.

\end{document}
